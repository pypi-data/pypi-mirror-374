\documentclass{galahad}

% set the release and package names

\newcommand{\package}{bsc}
\newcommand{\packagename}{BSC}
\newcommand{\fullpackagename}{\libraryname\_\packagename}

\begin{document}

\galheader

%%%%%%%%%%%%%%%%%%%%%% SUMMARY %%%%%%%%%%%%%%%%%%%%%%%%

\galsummary
Given matrices $\bmA$ and diagonal $\bmD$, this package forms the 
{\bf Schur complement} $\bmS = \bmA \bmD \bmA^T$ in sparse co-ordinate format.
Full advantage is taken of any zero coefficients in the matrices $\bmA$.

%%%%%%%%%%%%%%%%%%%%%% attributes %%%%%%%%%%%%%%%%%%%%%%%%

\galattributes
\galversions{\tt  \fullpackagename\_single, \fullpackagename\_double}.
\galuses {\tt GALAHAD\_CLOCK},
{\tt GALAHAD\_\-SY\-M\-BOLS}, 
{\tt GALAHAD\-\_SPACE}, 
{\tt GALAHAD\_SMT},
{\tt GALAHAD\_QPT},
{\tt GALAHAD\_SPECFILE},
\galdate October 2013.
\galorigin N. I. M. Gould,
Rutherford Appleton Laboratory.
\gallanguage Fortran~95 + TR 15581 or Fortran~2003. 
%\galparallelism Some options may use OpenMP and its runtime library.

%%%%%%%%%%%%%%%%%%%%%% HOW TO USE %%%%%%%%%%%%%%%%%%%%%%%%

\galhowto

%\subsection{Calling sequences}

The package is available with single, double and (if available) 
quadruple precision reals, and either 32-bit or 64-bit integers. 
Access to the 32-bit integer,
single precision version requires the {\tt USE} statement
\medskip

\hspace{8mm} {\tt USE \fullpackagename\_single}

\medskip
\noindent
with the obvious substitution 
{\tt \fullpackagename\_double},
{\tt \fullpackagename\_quadruple},
{\tt \fullpackagename\_single\_64},
{\tt \fullpackagename\_double\_64} and
{\tt \fullpackagename\_quadruple\_64} for the other variants.


\noindent
If it is required to use more than one of the modules at the same time, 
the derived types
{\tt SMT\_type}, 
{\tt QPT\_problem\_type}, 
{\tt \packagename\_time\_type}, 
{\tt \packagename\_control\_type}, 
{\tt \packagename\_inform\_type} 
and
{\tt \packagename\_data\_type}
(\S\ref{galtypes})
and the subroutines
{\tt \packagename\_initialize}, 
{\tt \packagename\_\-form},
{\tt \packagename\_terminate},
(\S\ref{galarguments})
and 
{\tt \packagename\_read\_specfile}
(\S\ref{galfeatures})
must be renamed on one of the {\tt USE} statements.

%%%%%%%%%%%%%%%%%%%%%% matrix formats %%%%%%%%%%%%%%%%%%%%%%%%

\galmatrix
The input matrix $\bmA$ may be stored in a variety of input formats.

\subsubsection{Dense storage format}\label{dense}
The matrix $\bmA$ is stored as a compact 
dense matrix by rows, that is, the values of the entries of each row in turn are
stored in order within an appropriate real one-dimensional array.
Component $n \ast (i-1) + j$ of the storage array {\tt A\%val} will hold the 
value $a_{ij}$ for $i = 1, \ldots , m$, $j = 1, \ldots , n$.

\subsubsection{Sparse co-ordinate storage format}\label{coordinate}
Only the nonzero entries of the matrices are stored. For the 
$l$-th entry of $\bmA$, its row index $i$, column index $j$ 
and value $a_{ij}$
are stored in the $l$-th components of the integer arrays {\tt A\%row}, 
{\tt A\%col} and real array {\tt A\%val}, respectively.
The order is unimportant, but the total
number of entries {\tt A\%ne} is also required. 

\subsubsection{Sparse row-wise storage format}\label{rowwise}
Again only the nonzero entries are stored, but this time
they are ordered so that those in row $i$ appear directly before those
in row $i+1$. For the $i$-th row of $\bmA$, the $i$-th component of a 
integer array {\tt A\%ptr} holds the position of the first entry in this row,
while {\tt A\%ptr} $(m+1)$ holds the total number of entries plus one.
The column indices $j$ and values $a_{ij}$ of the entries in the $i$-th row 
are stored in components 
$l =$ {\tt A\%ptr}$(i)$, \ldots ,{\tt A\%ptr} $(i+1)-1$ of the 
integer array {\tt A\%col}, and real array {\tt A\%val}, respectively. 

For sparse matrices, this scheme almost always requires less storage than 
its predecessor.

%%%%%%%%%%%%%%%%%%%%%%%%%%% kinds %%%%%%%%%%%%%%%%%%%%%%%%%%%

%%%%%%%%%%%%%%%%%%%%%% kinds %%%%%%%%%%%%%%%%%%%%%%

\galkinds
We use the terms integer and real to refer to the fortran
keywords {\tt REAL(rp\_)} and {\tt INTEGER(ip\_)}, where 
{\tt rp\_} and {\tt ip\_} are the relevant kind values for the
real and integer types employed by the particular module in use. 
The former are equivalent to
default {\tt REAL} for the single precision versions,
{\tt DOUBLE PRECISION} for the double precision cases
and quadruple-precision if 128-bit reals are available, and 
correspond to {\tt rp\_ = real32}, {\tt rp\_ = real64} and
{\tt rp\_ = real128} respectively as defined by the 
fortran {\tt iso\_fortran\_env} module.
The latter are default (32-bit) and long (64-bit) integers, and
correspond to {\tt ip\_ = int32} and {\tt ip\_ = int64},
respectively, again from the {\tt iso\_fortran\_env} module.


%%%%%%%%%%%%%%%%%%%%%% parallel usage %%%%%%%%%%%%%%%%%%%%%%%%

%%%%%%%%%%%%%%%%%%%%%%% parallel usage %%%%%%%%%%%%%%%%%%%%%%%%

\subsection{Parallel usage}
OpenMP may be used by the {\tt \fullpackagename} package to provide
parallelism for some solvers in shared memory environments.
See the documentation for the \galahad\ package {\tt SLS} for more details.
To run in parallel, OpenMP
must be enabled at compilation time by using the correct compiler flag
(usually some variant of {\tt -openmp}).
The number of threads may be controlled at runtime
by setting the environment variable {\tt OMP\_NUM\_THREADS}.

\noindent
MPI may also be used by the package to provide
parallelism for some solvers in a distributed memory environment.
To use this form of parallelism, MPI must be enabled at runtime 
by using the correct compiler flag
(usually some variant of {\tt -lmpi}).
Although the MPI process will be started automatically when required,
it should be stopped by the calling program once no further use of
this form of parallelism is needed. Typically, this will be via
statements of the form
\begin{verbatim}
      CALL MPI_INITIALIZED( flag, ierr )
      IF ( flag ) CALL MPI_FINALIZE( ierr )
\end{verbatim}

\noindent
The code may be compiled and run in serial mode.


%%%%%%%%%%%%%%%%%%%%%% derived types %%%%%%%%%%%%%%%%%%%%%%%%

\galtypes
Four derived data types are accessible from the package.

%%%%%%%%%%% matrix data type %%%%%%%%%%%

\subsubsection{The derived data type for holding matrices}\label{typesmt}
The derived data type {\tt SMT\_TYPE} is used to hold the matrix $\bmA$.
The components of {\tt SMT\_TYPE} used here are:

\begin{description}

%\ittf{m} is a scalar component of type \integer, 
%that holds the number of rows in the matrix. 
 
%\ittf{n} is a scalar component of type \integer, 
%that holds the number of columns in the matrix. 
 
\ittf{type} is a rank-one allocatable array of type default \character, that
is used to indicate the storage scheme used. If the dense storage scheme 
(see \S\ref{dense}), is used, 
the first five components of {\tt type} must contain the
string {\tt DENSE}.
For the sparse co-ordinate scheme (see \S\ref{coordinate}), 
the first ten components of {\tt type} must contain the
string {\tt COORDINATE},  and
for the sparse row-wise storage scheme (see \S\ref{rowwise}),
the first fourteen components of {\tt type} must contain the
string {\tt SPARSE\_BY\_ROWS}.

For convenience, the procedure {\tt SMT\_put} 
may be used to allocate sufficient space and insert the required keyword
into {\tt type}.
For example, if {\tt A} is of derived type {\tt SMT\_type}
and we wish to use the co-ordinate storage scheme, we may simply
%\vspace*{-2mm}
{\tt 
\begin{verbatim}
        CALL SMT_put( A%type, 'COORDINATE', istat )
\end{verbatim}
}
%\vspace*{-4mm}
\noindent
See the documentation for the \galahad\ package {\tt SMT} 
for further details on the use of {\tt SMT\_put}.

\ittf{ne} is a scalar variable of type \integer, that
holds the number of matrix entries.

\ittf{val} is a rank-one allocatable array of type \realdp\, 
and dimension at least {\tt ne}, that holds the values of the entries. 
Any duplicated entries that appear in the sparse 
co-ordinate or row-wise schemes will be summed. 

\ittf{row} is a rank-one allocatable array of type \integer, 
and dimension at least {\tt ne}, that may hold the row indices of the entries. 
(see \S\ref{coordinate}).

\ittf{col} is a rank-one allocatable array of type \integer, 
and dimension at least {\tt ne}, that may hold the column indices of the entries
(see \S\ref{coordinate}--\ref{rowwise}).

\ittf{ptr} is a rank-one allocatable array of type \integer, 
and dimension at least {\tt m + 1}, that may hold the pointers to
the first entry in each row (see \S\ref{rowwise}).

\end{description}


%%%%%%%%%%% control type %%%%%%%%%%%

\subsubsection{The derived data type for holding control 
 parameters}\label{typecontrol}
The derived data type 
{\tt \packagename\_control\_type} 
is used to hold controlling data. Default values may be obtained by calling 
{\tt \packagename\_initialize}
(see \S\ref{subinit}),
while components may also be changed by calling 
{\tt \fullpackagename\_read\-\_spec}
(see \S\ref{readspec}). 
The components of 
{\tt \packagename\_control\_type} 
are:

\begin{description}

\itt{error} is a scalar variable of type \integer, that holds the
stream number for error messages. Printing of error messages in 
{\tt \packagename\_solve} and {\tt \packagename\_terminate} 
is suppressed if {\tt error} $\leq 0$.
The default is {\tt error = 6}.

\ittf{out} is a scalar variable of type \integer, that holds the
stream number for informational messages. Printing of informational messages in 
{\tt \packagename\_solve} is suppressed if {\tt out} $< 0$.
The default is {\tt out = 6}.

\itt{print\_level} is a scalar variable of type \integer, that is used
to control the amount of informational output which is required. No 
informational output will occur if {\tt print\_level} $\leq 0$. If 
{\tt print\_level} $= 1$, a single line of output will be produced for each
iteration of the process. If {\tt print\_level} $\geq 2$, this output will be
increased to provide significant detail of each iteration.
The default is {\tt print\_level = 0}.

\itt{new\_a} is a scalar variable of type \integer, that is used
to indicate how $\bmA$ has changed (if at all) since the previous call
to {\tt BSC\_form}. Possible values are:
\begin{description}
\itt{0} $\bmA$ is unchanged
\itt{1} the values in $\bmA$ have changed, but its nonzero structure 
is as before.
\itt{2} both the values and structure of $\bmA$ have changed.
\itt{3} the structure of $\bmA$ has changed, but only the structure of
$\bmS$ (and not its values) is required.
\end{description}
The default is {\tt new\_a = 2}.

\itt{max\_col} is a scalar variable of type \integer, that specifies
the maximum number of nonzeros in a column of $\bmA$ which is permitted
when building the Schur-complement. Any negative value will be interpreted
as {\tt m}.
The default is {\tt max\_col = -1}.

\itt{extra\_space\_s} is a scalar variable of type \integer, 
that specifies how much extra space (if any) is to be allocated for the arrays 
that will hold $\bmS$ above that needed to hold the Schur complement.
This may be useful, for example, if another matrix might be subsequently added
to $\bmS$.
The default is {\tt extra\_space\_s = 0}.

\itt{s\_also\_by\_column} is a scalar variable of type default \logical, 
that must be set \true\ if the array {\tt S\%ptr} should be allocated and set 
to indicate the first entry in each column of $\bmS$, and  \false\ otherwise. 
The default is {\tt s\_also\_by\_column = .FALSE.}.

\itt{space\_critical} is a scalar variable of type default \logical, 
that must be set \true\ if space is critical when allocating arrays
and  \false\ otherwise. The package may run faster if 
{\tt space\_critical} is \false\ but at the possible expense of a larger
storage requirement. The default is {\tt space\_critical = .FALSE.}.

\itt{deallocate\_error\_fatal} is a scalar variable of type default \logical, 
that must be set \true\ if the user wishes to terminate execution if
a deallocation  fails, and \false\ if an attempt to continue
will be made. The default is {\tt deallocate\_error\_fatal = .FALSE.}.

\itt{prefix} is a scalar variable of type default \character\
and length 30, that may be used to provide a user-selected 
character string to preface every line of printed output. 
Specifically, each line of output will be prefaced by the string 
{\tt prefix(2:LEN(TRIM(prefix))-1)},
thus ignoring the first and last non-null components of the
supplied string. If the user does not want to preface lines by such
a string, they may use the default {\tt prefix = ""}.

\end{description}

%%%%%%%%%%% inform type %%%%%%%%%%%

\subsubsection{The derived data type for holding informational
 parameters}\label{typeinform}
The derived data type 
{\tt \packagename\_inform\_type} 
is used to hold parameters that give information about the progress and needs 
of the algorithm. The components of 
{\tt \packagename\_inform\_type} 
are:

\begin{description}

\itt{status} is a scalar variable of type \integer, that gives the
exit status of the algorithm. 
%See Sections~\ref{galerrors} and \ref{galinfo}
See \S\ref{galerrors} 
for details.

\itt{alloc\_status} is a scalar variable of type \integer, that gives
the status of the last attempted array allocation or deallocation.
This will be 0 if {\tt status = 0}.

\itt{bad\_alloc} is a scalar variable of type default \character\
and length 80, that  gives the name of the last internal array 
for which there were allocation or deallocation errors.
This will be the null string if {\tt status = 0}. 

\itt{max\_col\_a} is a scalar variable of type \integer, that gives
 the maximum number of entries in a column of $\bmA$.

\itt{exceeds\_max\_col} is a scalar variable of type \integer, that 
 gives the number of columns of $\bmA$ that have more entries than the limit
 specified by {\tt control\%max\_col}.

\itt{time} is a scalar variable of type \realdp, that gives
 the total CPU time (in seconds) spent in the package.

\itt{clock\_time} is a scalar variable of type \realdp, that gives
 the total elapsed system clock time (in seconds) spent in the package.

\end{description}

%%%%%%%%%%% data type %%%%%%%%%%%

\subsubsection{The derived data type for holding problem data}\label{typedata}
The derived data type
{\tt \packagename\_data\_type} 
is used to hold all the data for the problem and the workspace arrays 
used to construct the Schur complement between calls of 
{\tt \packagename} procedures. 
This data should be preserved, untouched, from the initial call to 
{\tt \packagename\_initialize}
to the final call to
{\tt \packagename\_terminate}.

%%%%%%%%%%%%%%%%%%%%%% argument lists %%%%%%%%%%%%%%%%%%%%%%%%

\galarguments
There are three procedures for user calls
(see \S\ref{galfeatures} for further features): 

\begin{enumerate}
\item The subroutine 
      {\tt \packagename\_initialize} 
      is used to set default values, and initialize private data, 
      before solving one or more problems with the
      same sparsity and bound structure.
\item The subroutine 
      {\tt \packagename\_form} 
      is called to form the Schur complement.
\item The subroutine 
      {\tt \packagename\_terminate} 
      is provided to allow the user to automatically deallocate array 
       components of the private data, allocated by 
       {\tt \packagename\_form} 
       at the end of the solution process. 
\end{enumerate}
We use square brackets {\tt [ ]} to indicate \optional arguments.

%%%%%% initialization subroutine %%%%%%

\subsubsection{The initialization subroutine}\label{subinit}
 Default values are provided as follows:
\vspace*{1mm}

\hspace{8mm}
{\tt CALL \packagename\_initialize( data, control, inform )}

\vspace*{-3mm}
\begin{description}

\itt{data} is a scalar \intentinout\ argument of type 
{\tt \packagename\_data\_type}
(see \S\ref{typedata}). It is used to hold data about the problem being 
solved. 

\itt{control} is a scalar \intentout\ argument of type 
{\tt \packagename\_control\_type}
(see \S\ref{typecontrol}). 
On exit, {\tt control} contains default values for the components as
described in \S\ref{typecontrol}.
These values should only be changed after calling 
{\tt \packagename\_initialize}.

\itt{inform} is a scalar \intentout\ argument of type 
{\tt \packagename\_inform\_type}
(see Section~\ref{typeinform}). A successful call to
{\tt \packagename\_initialize}
is indicated when the  component {\tt status} has the value 0. 
For other return values of {\tt status}, see Section~\ref{galerrors}.

\end{description}

%%%%%%%%% main solution subroutine %%%%%%

\subsubsection{The subroutine for forming the Schur complement}
The sparse matrix $\bmS = \bmA \bmD \bmA^T$ is formed as follows:
\vspace*{1mm}

\hspace{8mm}
{\tt CALL \packagename\_form( m, n, A, S, data, control, inform[, D] )}

%\vspace*{-3mm}
\begin{description}
\itt{m} is a scalar \intentin\ argument of type \integer\ that specifies
the number of rows of $\bmA$. 
\restriction {\tt m} $\geq 0$.

\itt{n} is a scalar \intentin\ argument of type \integer\ that specifies
the number of columns of $\bmA$.
\restriction {\tt n} $> 0$.

\itt{A} is a scalar \intentin\ argument of type {\tt SMT\_type} whose
components must be set to specify the data defining the matrix $\bmA$ 
(see \S\ref{typesmt}).

\itt{S} is a scalar \intentout\ argument of type {\tt SMT\_type} whose
components will be set to specify the {\em lower triangle} of the 
Schur complement $\bmS = \bmA \bmD \bmA^T$. In particular, 
the nonzeros of the lower triangle of $\bmS$ will be returned
in co-ordinate form (see \S\ref{coordinate}). Specifically
{\tt S\%type} contains the string 
{\tt COORDINATE}, {\tt S\%ne} gives the number of nonzeros,
and the array entries {\tt S\%row(i)}, {\tt S\%col(i)} and 
%(if {\tt control\%extra\_space\_s} $\leq 2$) 
{\tt S\%val(i)}, {\tt i = 1, \ldots, S\%ne} give row and column
indices and values of the entries in the lower triangle of $\bmS$
(see \S\ref{typesmt}). In addition, for compatibility with other
\galahad\ packages, {\tt S\%m} and {\tt S\%n} provide the row and column
dimensions, $m$ and $n$, of $\bmS$.

\itt{data} is a scalar \intentinout\ argument of type 
{\tt \packagename\_data\_type}
(see \S\ref{typedata}). It is used to hold data about the problem being 
solved. It must not have been altered {\bf by the user} since the last call to 
{\tt \packagename\_initialize}.

\itt{control} is a scalar \intentin\ argument of type 
{\tt \packagename\_control\_type}
(see \S\ref{typecontrol}). Default values may be assigned by calling 
{\tt \packagename\_initialize} prior to the first call to 
{\tt \packagename\_solve}.

\itt{inform} is a scalar \intentout\ argument of type 
{\tt \packagename\_inform\_type}
(see \S\ref{typeinform}). A successful call to
{\tt \packagename\_solve}
is indicated when the  component {\tt status} has the value 0. 
For other return values of {\tt status}, see \S\ref{galerrors}.

\itt{D} is a rank-one \optional\ \intentin\ argument of type \realdp\
and length at least {\tt n}, whose $i$-th component give the value of the 
$i$-th diagonal entry of the matrix $\bmD$. If {\tt D} is absent, $\bmD$
will be assumed to be the identity matrix.

\end{description}

%%%%%%% termination subroutine %%%%%%

\subsubsection{The  termination subroutine}
All previously allocated arrays are deallocated as follows:
\vspace*{1mm}

\hspace{8mm}
{\tt CALL \packagename\_terminate( data, control, inform )}

%\vspace*{-3mm}
\begin{description}

\itt{data} is a scalar \intentinout\ argument of type 
{\tt \packagename\_data\_type} 
exactly as for
{\tt \packagename\_solve},
which must not have been altered {\bf by the user} since the last call to 
{\tt \packagename\_initialize}.
On exit, array components will have been deallocated.

\itt{control} is a scalar \intentin\ argument of type 
{\tt \packagename\_control\_type}
exactly as for
{\tt \packagename\_solve}.

\itt{inform} is a scalar \intentout\ argument of type
{\tt \packagename\_inform\_type}
exactly as for
{\tt \packagename\_solve}.
Only the component {\tt status} will be set on exit, and a 
successful call to 
{\tt \packagename\_terminate}
is indicated when this  component {\tt status} has the value 0. 
For other return values of {\tt status}, see \S\ref{galerrors}.

\end{description}

%%%%%%%%%%%%%%%%%%%%%% Warning and error messages %%%%%%%%%%%%%%%%%%%%%%%%

\galerrors
A negative value of {\tt inform\%status} on exit from 
{\tt \packagename\_form}, 
{\tt \packagename\_solve}
or 
{\tt \packagename\_terminate}
indicates that an error has occurred. No further calls should be made
until the error has been corrected. Possible values are:

\begin{description}

\itt{\galerrallocate.} An allocation error occurred. 
A message indicating the offending 
array is written on unit {\tt control\%error}, and the returned allocation 
status and a string containing the name of the offending array
are held in {\tt inform\%alloc\_\-status}
and {\tt inform\%bad\_alloc} respectively.

\itt{\galerrdeallocate.} A deallocation error occurred. 
A message indicating the offending 
array is written on unit {\tt control\%error} and the returned allocation 
status and a string containing the name of the offending array
are held in {\tt inform\%alloc\_\-status}
and {\tt inform\%bad\_alloc} respectively.

\itt{\galerrrestrictions.} One of the restrictions 
   {\tt n} $> 0$ or {\tt m} $\geq  0$
    or requirements that {\tt prob\%A\_type}
    contain the string
    {\tt 'DENSE'}, {\tt 'COORDINATE'} or {\tt 'SPARSE\_BY\_ROWS'}
    has been violated.

\itt{\galerrschurcomplement}
A row of $\bmA$ has more than {\tt control\%max\_col} entries.

\end{description}

%%%%%%%%%%%%%%%%%%%%%% Further features %%%%%%%%%%%%%%%%%%%%%%%%

\galfeatures
\noindent In this section, we describe an alternative means of setting 
control parameters, that is components of the variable {\tt control} of type
{\tt \packagename\_control\_type}
(see \S\ref{typecontrol}), 
by reading an appropriate data specification file using the
subroutine {\tt \packagename\_read\_specfile}. This facility
is useful as it allows a user to change  {\tt \packagename} control parameters 
without editing and recompiling programs that call {\tt \packagename}.

A specification file, or specfile, is a data file containing a number of 
"specification commands". Each command occurs on a separate line, 
and comprises a "keyword", 
which is a string (in a close-to-natural language) used to identify a 
control parameter, and 
an (optional) "value", which defines the value to be assigned to the given
control parameter. All keywords and values are case insensitive, 
keywords may be preceded by one or more blanks but
values must not contain blanks, and
each value must be separated from its keyword by at least one blank.
Values must not contain more than 30 characters, and 
each line of the specfile is limited to 80 characters,
including the blanks separating keyword and value.



The portion of the specification file used by 
{\tt \packagename\_read\_specfile}
must start
with a "{\tt BEGIN \packagename}" command and end with an 
"{\tt END}" command.  The syntax of the specfile is thus defined as follows:
\begin{verbatim}
  ( .. lines ignored by BSC_read_specfile .. )
    BEGIN BSC
       keyword    value
       .......    .....
       keyword    value
    END 
  ( .. lines ignored by BSC_read_specfile .. )
\end{verbatim}
where keyword and value are two strings separated by (at least) one blank.
The ``{\tt BEGIN \packagename}'' and ``{\tt END}'' delimiter command lines 
may contain additional (trailing) strings so long as such strings are 
separated by one or more blanks, so that lines such as
\begin{verbatim}
    BEGIN BSC SPECIFICATION
\end{verbatim}
and
\begin{verbatim}
    END BSC SPECIFICATION
\end{verbatim}
are acceptable. Furthermore, 
between the
``{\tt BEGIN \packagename}'' and ``{\tt END}'' delimiters,
specification commands may occur in any order.  Blank lines and
lines whose first non-blank character is {\tt !} or {\tt *} are ignored. 
The content 
of a line after a {\tt !} or {\tt *} character is also 
ignored (as is the {\tt !} or {\tt *}
character itself). This provides an easy manner to "comment out" some 
specification commands, or to comment specific values 
of certain control parameters.  

The value of a control parameters may be of three different types, namely
integer, logical or real.
Integer and real values may be expressed in any relevant Fortran integer and
floating-point formats (respectively). Permitted values for logical
parameters are "{\tt ON}", "{\tt TRUE}", "{\tt .TRUE.}", "{\tt T}", 
"{\tt YES}", "{\tt Y}", or "{\tt OFF}", "{\tt NO}",
"{\tt N}", "{\tt FALSE}", "{\tt .FALSE.}" and "{\tt F}". 
Empty values are also allowed for 
logical control parameters, and are interpreted as "{\tt TRUE}".  

The specification file must be open for 
input when {\tt \packagename\_read\_specfile}
is called, and the associated device number 
passed to the routine in device (see below). 
Note that the corresponding 
file is {\tt REWIND}ed, which makes it possible to combine the specifications 
for more than one program/routine.  For the same reason, the file is not
closed by {\tt \packagename\_read\_specfile}.

\subsubsection{To read control parameters from a specification file}
\label{readspec}

Control parameters may be read from a file as follows:
\hskip0.5in 

\def\baselinestretch{0.8}
{\tt 
\begin{verbatim}
     CALL BSC_read_specfile( control, device )
\end{verbatim}
}
\def\baselinestretch{1.0}

\begin{description}
\itt{control} is a scalar \intentinout argument of type 
{\tt \packagename\_control\_type}
(see \S\ref{typecontrol}). 
Default values should have already been set, perhaps by calling 
{\tt \packagename\_initialize}.
On exit, individual components of {\tt control} may have been changed
according to the commands found in the specfile. Specfile commands and 
the component (see \S\ref{typecontrol}) of {\tt control} 
that each affects are given in Table~\ref{specfile}.
\bctable{|l|l|l|} 
\hline
  command & component of {\tt control} & value type \\ 
\hline

  {\tt error-printout-device} & {\tt \%error} & integer \\
  {\tt printout-device} & {\tt \%out} & integer \\
  {\tt print-level} & {\tt \%print\_level} & integer \\
  {\tt has-a-changed}   & {\tt \%new\_a} & integer \\
  {\tt maximum-column-nonzeros-in-schur-complement}  & {\tt \%max\_col} & integer \\
  {\tt extra-space-in-s}  & {\tt \%extra\_space\_s} & integer \\
  {\tt space-critical}   & {\tt \%space\_critical} & logical \\
  {\tt deallocate-error-fatal}   & {\tt \%deallocate\_error\_fatal} & logical \\
  {\tt output-line-prefix} & {\tt \%prefix} & character \\
\hline

\ectable{\label{specfile}Specfile commands and associated 
components of {\tt control}.}

%Individual components of the components {\tt control\%SLS\_control} 
%and {\tt control\%ULS\_control} 
%may be changed by specifying separate specfiles of the form
%\begin{verbatim}
%    BEGIN SLS
%     .......
%    END 
%\end{verbatim}
%and
%\begin{verbatim}
%    BEGIN ULS
%     .......
%%    END 
%\end{verbatim}
%just as described in the documentation for the \galahad\ packages 
%{\tt SLS} and {\tt ULS}.

\itt{device} is a scalar \intentin argument of type \integer,
that must be set to the unit number on which the specfile
has been opened. If {\tt device} is not open, {\tt control} will
not be altered and execution will continue, but an error message
will be printed on unit {\tt control\%error}.

\end{description}

%%%%%%%%%%%%%%%%%%%%%% Information printed %%%%%%%%%%%%%%%%%%%%%%%%

\galinfo
If {\tt control\%print\_level} is positive, information about the progress 
of the algorithm will be printed on unit {\tt control\-\%out}.
If {\tt control\%print\_level} $\geq 1$, statistics concerning the 
formation of $\bmS$
as well as warning and error messages will be reported. 

%%%%%%%%%%%%%%%%%%%%%% GENERAL INFORMATION %%%%%%%%%%%%%%%%%%%%%%%%

\galgeneral

\galcommon None.
\galworkspace Provided automatically by the module.
\galroutines None. 
\galmodules {\tt \packagename\_solve} calls the \galahad\ packages
{\tt GALAHAD\_CLOCK},
{\tt GALAHAD\_SY\-M\-BOLS}, \\
{\tt GALAHAD\-\_SPACE},
{\tt GALAHAD\_SMT},
{\tt GALAHAD\_QPT} and
{\tt GALAHAD\_SPECFILE},
\galio Output is under control of the arguments
 {\tt control\%error}, {\tt control\%out} and {\tt control\%print\_level}.
\galrestrictions {\tt n} $> 0$, {\tt m} $\geq  0$, 
{\tt A\_type} $\in \{${\tt 'DENSE'}, 
 {\tt 'COORDINATE'}, {\tt 'SPARSE\_BY\_ROWS'} $\}$. 
\galportability ISO Fortran~95 + TR 15581 or Fortran~2003. 
The package is thread-safe.

\end{description}

%%%%%%%%%%%%%%%%%%%%%% EXAMPLE %%%%%%%%%%%%%%%%%%%%%%%%

\galexample
Suppose we form the Schur complement $\bmS = \bmA \bmD \bmA^T$ with matrix data
\disp{ \bmA = \mat{cccc}{ 1 & 1 & \\ & & 1 & 1 \\ 1 & & & 1}
\tim{and}  \bmD = \mat{cccc}{ 1 \\ & 2 \\ & & 3 \\ & & & 4 }}
Then storing the matrices in sparse co-ordinate format,
we may use the following code:

{\tt \small
\VerbatimInput{\packageexample}
}
\noindent
This produces the following output:
{\tt \small
\VerbatimInput{\packageresults}
}
\noindent
The same problem may be solved holding the data in 
a sparse row-wise storage format by replacing the lines
{\tt \small
\begin{verbatim}
!  sparse co-ordinate storage format
...
! problem data complete   
\end{verbatim}
}
\noindent
by
{\tt \small
\begin{verbatim}
! sparse row-wise storage format
   CALL SMT_put( A%type, 'SPARSE_BY_ROWS', i ) ! storage for A
   ALLOCATE( A%val( a_ne ), A%col( a_ne ), A%ptr( m + 1 ) )
   A%val = (/ 1.0_wp, 1.0_wp, 1.0_wp, 1.0_wp, 1.0_wp, 1.0_wp /) ! matrix A
   A%col = (/ 1, 2, 3, 4, 1, 4 /)
   A%ptr = (/ 1, 3, 5, 7 /)                 ! Set row pointers  
! problem data complete   
\end{verbatim}
}
\noindent
or using a dense storage format with the replacement lines
{\tt \small
\begin{verbatim}
! dense storage format
   CALL SMT_put( A%type, 'DENSE', i )  ! storage for A
   ALLOCATE( A%val( n * m ) )
   A%val = (/ 1.0_wp, 1.0_wp, 0.0_wp, 0.0_wp, 0.0_wp, 0.0_wp,          &
              1.0_wp, 1.0_wp, 1.0_wp, 0.0_wp, 0.0_wp, 1.0_wp /) ! A
! problem data complete   
\end{verbatim}
}
\noindent
respectively.

\end{document}

