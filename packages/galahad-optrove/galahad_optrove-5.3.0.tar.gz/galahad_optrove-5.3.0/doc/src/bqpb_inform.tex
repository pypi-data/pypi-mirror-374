\begin{description}

\itt{status} is a scalar variable of type \integer, that gives the
exit status of the algorithm.
%See Sections~\ref{galerrors} and \ref{galinfo}
See Section~\ref{galerrors}
for details.

\itt{alloc\_status} is a scalar variable of type \integer, that gives
the status of the last attempted array allocation or deallocation.
This will be 0 if {\tt status = 0}.

\itt{bad\_alloc} is a scalar variable of type default \character\
and length 80, that  gives the name of the last internal array
for which there were allocation or deallocation errors.
This will be the null string if {\tt status = 0}.

\itt{factorization\_status} is a scalar variable of type \integer, that
gives the return status from the matrix factorization.

\itt{factorization\_integer} is a scalar variable of type long
\integer, that gives the amount of integer storage used for the matrix
factorization.

\itt{factorization\_real} is a scalar variable of type \longinteger,
that gives the amount of real storage used for the matrix factorization.

\itt{nfacts} is a scalar variable of type \integer, that gives the
total number of factorizations performed.

\itt{nbacts} is a scalar variable of type \integer, that gives the
total number of backtracks performed during the sequence of linesearches.

\itt{threads} is a scalar variable of type \integer, that gives the
total number of threads used for parallel execution.

\itt{iter} is a scalar variable of type \integer, that
gives the number of iterations performed.

\ittf{obj} is a scalar variable of type \realdp, that holds the
value of the objective function at the best estimate of the solution found.

%\itt{primal\_infeasibility} is a scalar variable of type \realdp,
%that holds the norm of the violation of primal optimality
%(see Section~\ref{typetime}) at the best estimate of the solution found.

\itt{dual\_infeasibility} is a scalar variable of type \realdp,
that holds the norm of the violation of dual optimality
(see Section~\ref{typetime}) at the best estimate of the solution found.

\itt{complementary\_slackness}
is a scalar variable of type \realdp,
that holds the norm of the violation of complementary slackness
(see Section~\ref{typetime}) at the best estimate of the solution found.

\itt{feasible} is a scalar variable of type default \logical, that has the
value \true\ if the output value of $\bmx$ satisfies the constraints,
and the value \false\ otherwise.

\ittf{time} is a scalar variable of type {\tt \packagename\_time\_type}
whose components are used to hold elapsed CPU and system clock times for the
various parts of the calculation (see Section~\ref{typetime}).

\itt{FDC\_inform} is a scalar variable of type
{\tt FDC\_inform\_type}
whose components are used to provide information about
any detection of linear dependencies
performed by the package
{\tt \libraryname\_FDC}.
See the specification sheet for the package
{\tt \libraryname\_FDC} for details.

\itt{SBLS\_inform} is a scalar variable of type
{\tt SBLS\_inform\_type}
whose components are used to provide information about factorizations
performed by the package
{\tt \libraryname\_SBLS}.
See the specification sheet for the package
{\tt \libraryname\_SBLS} for details.

\itt{FIT\_inform} is a scalar variable of type
{\tt FIT\_inform\_type}
whose components are used to provide information about the fitting
of data to polynomials performed by the package
{\tt \libraryname\_FIT}.
See the specification sheet for the package
{\tt \libraryname\_FIT} for details.

\itt{ROOTS\_inform} is a scalar variable of type
{\tt ROOTS\_inform\_type}
whose components are used to provide information about the
polynomial root finding performed by the package
{\tt \libraryname\_ROOTS}.
See the specification sheet for the package
{\tt \libraryname\_ROOTS} for details.

\itt{CRO\_inform} is a scalar variable of type
{\tt CRO\_inform\_type}
whose components are used to provide information about the crossover
performed by the package
{\tt \libraryname\_CRO}.
See the specification sheet for the package
{\tt \libraryname\_CRO} for details.

\end{description}
