\documentclass{galahad}

% set the package name

\newcommand{\package}{lstr}
\newcommand{\packagename}{LS\-TR}
\newcommand{\fullpackagename}{\libraryname\_\packagename}

\begin{document}

\galheader

%%%%%%%%%%%%%%%%%%%%%% SUMMARY %%%%%%%%%%%%%%%%%%%%%%%%

\galsummary
Given a real $m$ by $n$ matrix $\bmA$, a real
$m$ vector $\bmb$ and a scalar $\Delta>0$, this package finds an
{\bf approximate minimizer of $\| \bmA \bmx - \bmb\|_2$, where the vector
$\bmx$ is required to satisfy the ``trust-region''
constraint $\|\bmx\|_2 \leq  \Delta$.}
This problem commonly occurs as a trust-region subproblem in nonlinear
optimization calculations, and may be used to regularize the solution
of under-determined or ill-conditioned linear least-squares problems.
The method may be suitable for large $m$ and/or $n$ as no factorization
involving $\bmA$ is required. Reverse communication is used to obtain
matrix-vector products of the form $\bmu + \bmA \bmv$ and
$\bmv + \bmA^T \bmu$.

%%%%%%%%%%%%%%%%%%%%%% attributes %%%%%%%%%%%%%%%%%%%%%%%%

\galattributes
\galversions{\tt  \fullpackagename\_single, \fullpackagename\_double}.
\galuses
{\tt \libraryname\_SY\-M\-BOLS},
{\tt \libraryname\_SPACE}, {\tt \libraryname\_\-NORMS},
{\tt \libraryname\_ROOTS}, {\tt \libraryname\_SPECFILE},
{\tt *ROTG}.
\galdate November 2007.
\galorigin N. I. M. Gould, Oxford University and Rutherford Appleton Laboratory.
\gallanguage Fortran~95 + TR 15581 or Fortran~2003.

%%%%%%%%%%%%%%%%%%%%%% HOW TO USE %%%%%%%%%%%%%%%%%%%%%%%%

\galhowto

The package is available with single, double and (if available) 
quadruple precision reals, and either 32-bit or 64-bit integers. 
Access to the 32-bit integer,
single precision version requires the {\tt USE} statement
\medskip

\hspace{8mm} {\tt USE \fullpackagename\_single}

\medskip
\noindent
with the obvious substitution 
{\tt \fullpackagename\_double},
{\tt \fullpackagename\_quadruple},
{\tt \fullpackagename\_single\_64},
{\tt \fullpackagename\_double\_64} and
{\tt \fullpackagename\_quadruple\_64} for the other variants.


\noindent
If it is required to use more than one of the modules at the same time, 
the derived types
{\tt \packagename\_control\_type}, {\tt \packagename\_inform\_type},
{\tt \packagename\_data\_type},
(Section~\ref{galtypes})
and the subroutines
{\tt \packagename\_initialize},
{\tt \packagename\_solve}, {\tt \packagename\_terminate}
(Section~\ref{galarguments})
and
{\tt \packagename\_read\_specfile}
(Section~\ref{galfeatures})
must be renamed on one of the {\tt USE} statements.

%%%%%%%%%%%%%%%%%%%%%%%%%%% kinds %%%%%%%%%%%%%%%%%%%%%%%%%%%

%%%%%%%%%%%%%%%%%%%%%% kinds %%%%%%%%%%%%%%%%%%%%%%

\galkinds
We use the terms integer and real to refer to the fortran
keywords {\tt REAL(rp\_)} and {\tt INTEGER(ip\_)}, where 
{\tt rp\_} and {\tt ip\_} are the relevant kind values for the
real and integer types employed by the particular module in use. 
The former are equivalent to
default {\tt REAL} for the single precision versions,
{\tt DOUBLE PRECISION} for the double precision cases
and quadruple-precision if 128-bit reals are available, and 
correspond to {\tt rp\_ = real32}, {\tt rp\_ = real64} and
{\tt rp\_ = real128} respectively as defined by the 
fortran {\tt iso\_fortran\_env} module.
The latter are default (32-bit) and long (64-bit) integers, and
correspond to {\tt ip\_ = int32} and {\tt ip\_ = int64},
respectively, again from the {\tt iso\_fortran\_env} module.


%%%%%%%%%%%%%%%%%%%%%% derived types %%%%%%%%%%%%%%%%%%%%%%%%

\galtypes
Three derived data types are accessible from the package.

%%%%%%%%%%% control type %%%%%%%%%%%

\subsubsection{The derived data type for holding control
 parameters}\label{typecontrol}
The derived data type
{\tt \packagename\_control\_type}
is used to hold controlling data. Default values may be obtained by calling
{\tt \packagename\_initialize}
(see Section~\ref{subinit}). The components of
{\tt \packagename\_control\_type}
are:

\begin{description}
\itt{error} is a scalar variable of type \integer, that holds the
stream number for error messages.
Printing of error messages in
{\tt \packagename\_solve} and {\tt \packagename\_terminate}
is suppressed if ${\tt error} \leq {\tt 0}$.
The default is {\tt error = 6}.

\ittf{out} is a scalar variable of type \integer, that holds the
stream number for informational messages.
Printing of informational messages in
{\tt \packagename\_solve} is suppressed if ${\tt out} < {\tt 0}$.
The default is {\tt out = 6}.

\itt{print\_level} is a scalar variable of type \integer,
that is used
to control the amount of informational output which is required. No
informational output will occur if ${\tt print\_level} \leq {\tt 0}$. If
{\tt print\_level = 1} a single line of output will be produced for each
iteration of the process. If {\tt print\_level} $\geq$ {\tt 2} this output
will be increased to provide significant detail of each iteration.
The default is {\tt print\_level = 0}.

\itt{itmin} is a scalar variable of type \integer, that holds the
minimum number of iterations which will be performed by
{\tt \packagename\_solve}.
The default is {\tt itmin = -1}.

\itt{itmax} is a scalar variable of type \integer, that holds the
maximum number of iterations which will be allowed in
{\tt \packagename\_solve}.
If {\tt itmax} is set to a negative number, it will be reset by
{\tt \packagename\_solve} to $\max(m,n)+1$.
The default is {\tt itmax = -1}.

\itt{itmax\_on\_boundary} is a scalar variable of type \integer,
that holds the
maximum number of iterations that may be performed when the iterates
encounter the boundary of the constraint.
If {\tt itmax\_on\_boundary} is set to a negative number, it will be reset by
{\tt \packagename\_solve} to $\max(m,n)+1$.
The default is {\tt itmax\_on\_boundary = -1}.

\itt{bitmax} is a scalar variable of type \integer, that holds the
maximum number of Newton inner iterations which will be allowed for each
main iteration in {\tt \packagename\_solve}.
If {\tt bitmax} is set to a negative number, it will be reset by
{\tt \packagename\_solve} to $10$.
The default is {\tt bitmax = -1}.

\itt{extra\_vectors} is a scalar variable of type \integer,
that specifies the number of additional vectors of length $n$
that will be allocated to try to speed up the computation if the
constraint boundary  is encountered.
The default is {\tt extra\_vectors = 0}.

\itt{steihaug\_toint} is a scalar variable of type default \logical,
which must be set \true\ if the algorithm is required to stop at
the first point on the boundary of the constraint that is encountered,
and \false\  if the constraint boundary is to be investigated
further.  Setting {\tt steihaug\_toint} to \true\ can reduce the
amount of computation at the expense of obtaining a poorer estimate of
the solution.
The default is {\tt steihaug\_toint = \true}.

\itt{space\_critical} is a scalar variable of type default \logical, that
may be set \true\ if the user wishes the package to allocate as little
internal storage as possible, and \false\ otherwise. The package may
be more efficient if {\tt space\_critical} is set \false.
The default is {\tt space\_critical = \false}.

\itt{deallocate\_error\_fatal} is a scalar variable of type default \logical,
that may be set \true\ if the user wishes the package to return to the user
in the unlikely event that an internal array deallocation fails,
and \false\ if the package should be allowed to try to continue.
The default is {\tt deallocate\_error\_fatal = \false}.

\itt{stop\_relative} and {\tt stop\_absolute} are scalar variables of type
\realdp, that holds the
relative and absolute convergence tolerances (see Section~\ref{galmethod}).
The computed solution $\bmx$ is accepted by {\tt \packagename\_solve}
if the computed value of
$\|\bmA^T ( \bmA \bmx - \bmb ) + \lambda \bmx\|_2$
is less than or equal
to $\max (\| \bmA^T \bmb \|_2 \ast$ {\tt stop\_relative},
{\tt stop\_absolute}$)$, where $\lambda$ is an estimate of the Lagrange
multiplier associated with the trust-region constraint.
The defaults are {\tt stop\_relative = $\sqrt{u}$} and
{\tt stop\_absolute = 0.0},
where $u$ is {\tt EPSILON(1.0)} ({\tt EPSILON(1.0D0)} in
{\tt \fullpackagename\_double}).

\itt{fraction\_opt} is a scalar variable of type default
\realdp, that specifies the fraction
of the optimal value which is to be considered acceptable by the algorithm.
A negative value is considered to be zero, and a value of larger than one
is considered to be one. Reducing {\tt fraction\_opt} below one will result
in a reduction of the computation performed at the expense of an inferior
optimal value.
The default is {\tt fraction\_opt = 1.0}.

\itt{prefix} is a scalar variable of type default \character\
and length 30, that may be used to provide a user-selected
character string to preface every line of printed output.
Specifically, each line of output will be prefaced by the string
{\tt prefix(2:LEN(TRIM(prefix))-1)},
thus ignoring the first and last non-null components of the
supplied string. If the user does not want to preface lines by such
a string, they may use the default {\tt prefix = ""}.

\end{description}

%%%%%%%%%%% inform type %%%%%%%%%%%

\subsubsection{The derived data type for holding informational
 parameters}\label{typeinform}
The derived data type
{\tt \packagename\_inform\_type}
is used to hold parameters that give information about the progress and needs
of the algorithm. The components of
{\tt \packagename\_inform\_type}
are:

\begin{description}
\itt{status} is a scalar variable of type \integer, that gives the
current status of the algorithm. See Sections~\ref{galreverse} and
\ref{galerrors} for details.

\itt{alloc\_status} is a scalar variable of type \integer,
that gives the status of the last internal array allocation
or deallocation. This will be 0 if {\tt status = 0}.

\itt{bad\_alloc} is a scalar variable of type default \character\
and length 80, that  gives the name of the last internal array
for which there were allocation or deallocation errors.
This will be the null string if {\tt status = 0}.

\itt{multiplier} is a scalar variable of type default
\realdp, that holds the
value of the Lagrange multiplier $\lambda$ associated with the constraint.

\itt{x\_norm} is a scalar variable of type \realdp,
that holds the current value of $\|\bmx\|_2$.

\itt{r\_norm} is a scalar variable of type \realdp,
that holds the current value of $\|\bmA\bmx-\bmb\|_2$.

\itt{Atr\_norm} is a scalar variable of type \realdp,
that holds the current value of
$\|\bmA^T ( \bmA \bmx - \bm b ) + \lambda \bmx \|_2$.

\ittf{iter} is a scalar variable of type \integer, that holds the
current number of Lanczos vectors used.

\itt{iter\_pass2} is a scalar variable of type \integer, that holds the
current number of Lanczos vectors used in the second pass.

\end{description}

%%%%%%%%%%% data type %%%%%%%%%%%

\subsubsection{The derived data type for holding problem data}\label{typedata}
The derived data type
{\tt \packagename\_data\_type}
is used to hold all the data for a particular problem between calls of
{\tt \packagename} procedures.
This data should be preserved, untouched, from the initial call to
{\tt \packagename\_initialize}
to the final call to
{\tt \packagename\_terminate}.

%%%%%%%%%%%%%%%%%%%%%% argument lists %%%%%%%%%%%%%%%%%%%%%%%%

\galarguments
There are three procedures for user calls
(see Section~\ref{galfeatures} for further features):

\begin{enumerate}
\item The subroutine
      {\tt \packagename\_initialize}
      is used to set default values, and initialize private data.
\item The subroutine
      {\tt \packagename\_solve}
      is called repeatedly to solve the problem.
      On each exit, the user may be expected to provide additional
      information and, if necessary, re-enter the subroutine.
\item The subroutine
      {\tt \packagename\_terminate}
      is provided to allow the user to automatically deallocate array
       components of the private data, allocated by
       {\tt \packagename\_solve},
       at the end of the solution process.
       It is important to do this if the data object is re-used for another
       problem since {\tt \packagename\_initialize} cannot test for this
       situation,
       and any existing associated targets will subsequently become
       unreachable.
\end{enumerate}

%%%%%% initialization subroutine %%%%%%

\subsubsection{The initialization subroutine}\label{subinit}
 Default values are provided as follows:

\hskip0.5in
{\tt CALL \packagename\_initialize( data, control, inform )}

\begin{description}

\itt{data} is a scalar \intentinout argument of type
{\tt \packagename\_data\_type}
(see Section~\ref{typedata}). It is used to hold data about the problem being
solved.

\itt{control} is a scalar \intentout argument of type
{\tt \packagename\_control\_type}
(see Section~\ref{typecontrol}).
On exit, {\tt control} contains default values for the components as
described in Section~\ref{typecontrol}.
These values should only be changed after calling
{\tt \packagename\_initialize}.

\itt{inform} is a scalar \intentout\ argument of type
{\tt \packagename\_inform\_type}
(see Section~\ref{typeinform}). A successful call to
{\tt \packagename\_initialize}
is indicated when the  component {\tt status} has the value 0.
For other return values of {\tt status}, see Section~\ref{galerrors}.

\end{description}

%%%%%%%%% problem solution subroutine %%%%%%

\subsubsection{The optimization problem solution subroutine}
The optimization problem solution algorithm is called as follows:

\hskip0.5in
{\tt CALL \packagename\_solve( m, n, radius, X, U, V, data, control, inform )}

\begin{description}

\ittf{m} is a scalar \intentin\ argument of type \integer, that must be
set to the number of equations, $m$. {\bf Restriction: } $m > 0$.

\ittf{n} is a scalar \intentin\ argument of type \integer, that must be
set to the number of unknowns, $n$. {\bf Restriction: } $n > 0$.

\itt{radius} is a scalar \intentin\ variable of type default
\realdp,
that must be set on initial entry
to the value of the radius of the quadratic constraint, $\Delta$.
{\bf Restriction: } $\Delta > 0$.

\ittf{X} is an array \intentinout\ argument of dimension {\tt n} and
type \realdp,
that holds an estimate of the solution $\bmx$ of the linear system.
On initial entry, {\tt X} need not be set.
It must not be changed between entries.
On exit, {\tt X} contains the current best estimate of the solution.

\ittf{U} is an array \intentinout\ argument of dimension {\tt m} and
type \realdp,
that is used to hold left-Lanczos vectors used during the iteration.
On initial or restart entry, {\tt U} must contain the vector $\bmb$.
If {\tt inform\%status} = 2 or 4 on exit, {\tt U} must be reset
as directed by  {\tt inform\%status}; otherwise
it must be left unchanged.

\ittf{V} is an array \intentinout\ argument of dimension {\tt n} and
type \realdp,
that is used to hold left-Lanczos vectors used during the iteration.
It need not be set on initial or restart entry.
If {\tt inform\%status} = 3 on exit, {\tt V} must be reset
as directed by  {\tt inform\%status}; otherwise it must be left unchanged.

\itt{data} is a scalar \intentinout argument of type
{\tt \packagename\_data\_type}
(see Section~\ref{typedata}). It is used to hold data about the problem being
solved. It must not have been altered {\bf by the user} since the last call to
{\tt \packagename\_initialize}.

\itt{control} is a scalar \intentin\ argument of type
{\tt \packagename\_control\_type}.
(see Section~\ref{typecontrol}).
Default values may be assigned by calling {\tt \packagename\_initialize}
prior to the first call to {\tt \packagename\_solve}.

\itt{inform} is a scalar \intentinout argument of type
{\tt \packagename\_inform\_type}
(see Section~\ref{typeinform}).
On initial entry, the component {\tt status} must be set to {\tt 1},
while for a restart entry it must be set to {\tt 5}.
The remaining components need not be set.
A successful call to
{\tt \packagename\_solve}
is indicated when the  component {\tt status} has the value 0.
For other return values of {\tt status}, see Sections~\ref{galreverse}
and \ref{galerrors}.
\end{description}

%%%%%%% termination subroutine %%%%%%

\subsubsection{The  termination subroutine}
All previously allocated arrays are deallocated as follows:

\hskip0.5in
{\tt CALL \packagename\_terminate( data, control, inform )}

\begin{description}

\itt{data} is a scalar \intentinout argument of type
{\tt \packagename\_data\_type}
exactly as for
{\tt \packagename\_solve}
that must not have been altered {\bf by the user} since the last call to
{\tt \packagename\_initialize}.
On exit, array components will have been deallocated.

\itt{control} is a scalar \intentin argument of type
{\tt \packagename\_control\_type}
exactly as for
{\tt \packagename\_solve}.

\itt{inform} is a scalar \intentout argument of type {\tt \packagename\_type}
exactly as for
{\tt \packagename\_solve}.
Only the component {\tt status} will be set on exit, and a
successful call to
{\tt \packagename\_terminate}
is indicated when this  component {\tt status} has the value 0.
For other return values of {\tt status}, see Section~\ref{galerrors}.

\end{description}

%%%%%%%%%%%%%%%%%%%%%% Reverse communications %%%%%%%%%%%%%%%%%%%%%%%%

\galreverse
A positive value of {\tt inform\%status} on exit from
{\tt \packagename\_solve} indicates that the user needs to take appropriate
action before re-entering the subroutine. Possible values are:

\begin{description}

\itt{2.} The user must perform the operation
\disp{\bmu := \bmu + \bmA \bmv,}
and recall {\tt \packagename\_solve}.
The vectors $\bmu$ and $\bmv$ are available in the arrays {\tt U}
and {\tt V} respectively, and the result
$\bmu$ must overwrite the content of {\tt U}.
No argument except {\tt U} should be altered before recalling
{\tt \packagename\_solve}.

\itt{3.} The user must perform the operation
\disp{\bmv := \bmv + \bmA^T \bmu,}
and recall {\tt \packagename\_solve}.
The vectors $\bmu$ and $\bmv$ are available in the arrays {\tt U}
and {\tt V} respectively, and the result
$\bmv$ must overwrite the content of {\tt V}.
No argument except {\tt V} should be altered before recalling
{\tt \packagename\_solve}.

\itt{4.}  The user should reset {\tt U} to $\bmb$ and recall
{\tt \packagename\_solve}.
No argument except {\tt U} should be altered before recalling
{\tt \packagename\_solve}.

\end{description}

%%%%%%%%%%%%%%%%%%%%%% Warning and error messages %%%%%%%%%%%%%%%%%%%%%%%%

\galerrors
A negative value of  {\tt inform\%status} on exit from
{\tt \packagename\_solve}
or
{\tt \packagename\_terminate}
indicates that an error has occurred. No further calls should be made
until the error has been corrected. Possible values are:

\begin{description}
\itt{\galerrallocate.} An allocation error occurred. A message indicating
the offending
array is written on unit {\tt control\%error}, and the returned allocation
status and a string containing the name of the offending array
are held in {\tt inform\%alloc\_\-status}
and {\tt inform\%bad\_alloc} respectively.

\itt{\galerrdeallocate.} A deallocation error occurred.
A message indicating the offending
array is written on unit {\tt control\%error} and the returned allocation
status and a string containing the name of the offending array
are held in {\tt inform\%alloc\_\-status}
and {\tt inform\%bad\_alloc} respectively.

\itt{\galerrrestrictions.} ({\tt \packagename\_solve} only)
At least one of the restrictions
{\tt m > 0},
{\tt n > 0}
or
{\tt radius > 0}
has been violated.

\itt{\galerrmaxiterations.} ({\tt \packagename\_solve} only) More than
{\tt control\%itmax} iterations have been performed without obtaining
convergence.

\itt{\galerrinput.} ({\tt \packagename\_solve} only)  {\tt inform\%status} is
not $>$ 0 on entry.

\itt{\galwarnboundary.} ({\tt \packagename\_solve} only)
The constraint boundary has been
encountered when the input value of {\tt control\%steihaug\-\_\-toint} was
set \true. The
solution is unlikely to have achieved the accuracy required by
{\tt control\%stop\_rela\-tive} and {\tt control\%stop\_absolute}.

\end{description}

%%%%%%%%%%%%%%%%%%%%%% Weighted problems %%%%%%%%%%%%%%%%%%%%%%%%

\subsection{Weighted least-squares, scaled trust-regions and preconditioning}
The package may also be used to solve weighted least-squares problems
involving the objective $\| \bmW(\bmA\bmx-\bmb)\|_2$ and a scaled trust region
$\|\bmS\bmx\|_2 \leq \Delta$ simply by solving instead the problem
\disp{ \min \| \bar{\bmA} \bar{\bmx} - \bar{\bmb}\|_2 \;\;
 \mbox{subject to } \;\; \|\bar{\bmx}\|_2 \leq \Delta,}
where $\bar{\bmA} = \bmW \bmA \bmS^{-1}$ and
$\bar{\bmb} = \bmW \bmb$
and then recovering $\bmx = \bmS^{-1} \bar{\bmx}$. Note the implication here
that $\bmS$ must be non-singular.

Thus on initial entry ({\tt inform\%status} = 1) and re-entry
({\tt inform\%status} = 4), {\tt U} should contain $\bmW \bmb$,
while for {\tt inform\%status} = 2 and 3 entries, the operations
\disp{\bmu := \bmu + \bmW \bmA \bmS^{-1} \bmv \;\; \mbox{and} \;\;
\bmv := \bmv + \bmS^{-T} \bmA^T \bmW^T \bmu}
respectively, should be performed.

Note that the choice of $\bmW$ and $\bmS$ will effect the convergence of the
method, and thus good choices may be used to accelerate its convergence. This
is often known as preconditioning, but be aware that preconditioning changes
the norms that define the problem. Good preconditioners will cluster
the singular values of $\bar{\bmA}$ around a few distinct values, and ideally
(but usually unrealistically) all the singular values will be mapped to 1.

%%%%%%%%%%%%%%%%%%%%%% Re-entry %%%%%%%%%%%%%%%%%%%%%%%%

\subsection{Restart entry with a new value of $\Delta$}
It commonly happens that, having solved the problem for a particular value
of the radius $\Delta$, a user now wishes to solve the problem for a different
value of $\Delta$. Rather than restarting the calculation with
{\tt inform\%status = 1,} a useful approximation may be found
resetting {\tt radius} to the new required value and {\tt U} to $\bmb$,
and recalling {\tt \packagename\_solve}
with {\tt inform\%status = 5} and the remaining arguments unchanged.
This will determine the best solution within the Krylov space investigated
in the previous minimization.

%%%%%%%%%%%%%%%%%%%%%% Further features %%%%%%%%%%%%%%%%%%%%%%%%

\galfeatures
\noindent In this section, we describe an alternative means of setting
control parameters, that is components of the variable {\tt control} of type
{\tt \packagename\_control\_type}
(see Section~\ref{typecontrol}),
by reading an appropriate data specification file using the
subroutine {\tt \packagename\_read\_specfile}. This facility
is useful as it allows a user to change  {\tt \packagename} control parameters
without editing and recompiling programs that call {\tt \packagename}.

A specification file, or specfile, is a data file containing a number of
"specification commands". Each command occurs on a separate line,
and comprises a "keyword",
which is a string (in a close-to-natural language) used to identify a
control parameter, and
an (optional) "value", which defines the value to be assigned to the given
control parameter. All keywords and values are case insensitive,
keywords may be preceded by one or more blanks but
values must not contain blanks, and
each value must be separated from its keyword by at least one blank.
Values must not contain more than 30 characters, and
each line of the specfile is limited to 80 characters,
including the blanks separating keyword and value.

The portion of the specification file used by
{\tt \packagename\_read\_specfile}
must start
with a "{\tt BEGIN \packagename}" command and end with an
"{\tt END}" command.  The syntax of the specfile is thus defined as follows:
\begin{verbatim}
  ( .. lines ignored by LSTR_read_specfile .. )
    BEGIN LSTR
       keyword    value
       .......    .....
       keyword    value
    END
  ( .. lines ignored by LSTR_read_specfile .. )
\end{verbatim}
where keyword and value are two strings separated by (at least) one blank.
The ``{\tt BEGIN \packagename}'' and ``{\tt END}'' delimiter command lines
may contain additional (trailing) strings so long as such strings are
separated by one or more blanks, so that lines such as
\begin{verbatim}
    BEGIN LSTR SPECIFICATION
\end{verbatim}
and
\begin{verbatim}
    END LSTR SPECIFICATION
\end{verbatim}
are acceptable. Furthermore,
between the
``{\tt BEGIN \packagename}'' and ``{\tt END}'' delimiters,
specification commands may occur in any order.  Blank lines and
lines whose first non-blank character is {\tt !} or {\tt *} are ignored.
The content
of a line after a {\tt !} or {\tt *} character is also
ignored (as is the {\tt !} or {\tt *}
character itself). This provides an easy manner to "comment out" some
specification commands, or to comment specific values
of certain control parameters.

The value of a control parameters may be of three different types, namely
integer, logical or real.
Integer and real values may be expressed in any relevant Fortran integer and
floating-point formats (respectively). Permitted values for logical
parameters are "{\tt ON}", "{\tt TRUE}", "{\tt .TRUE.}", "{\tt T}",
"{\tt YES}", "{\tt Y}", or "{\tt OFF}", "{\tt NO}",
"{\tt N}", "{\tt FALSE}", "{\tt .FALSE.}" and "{\tt F}".
Empty values are also allowed for
logical control parameters, and are interpreted as "{\tt TRUE}".

The specification file must be open for
input when {\tt \packagename\_read\_specfile}
is called, and the associated device number
passed to the routine in device (see below).
Note that the corresponding
file is {\tt REWIND}ed, which makes it possible to combine the specifications
for more than one program/routine.  For the same reason, the file is not
closed by {\tt \packagename\_read\_specfile}.

\subsubsection{To read control parameters from a specification file}
\label{readspec}

Control parameters may be read from a file as follows:
\hskip0.5in
\def\baselinestretch{0.8} {\tt \begin{verbatim}
     CALL LSTR_read_specfile( control, device )
\end{verbatim}}
\def\baselinestretch{1.0}

\begin{description}
\itt{control} is a scalar \intentinout argument of type
{\tt \packagename\_control\_type}
(see Section~\ref{typecontrol}).
Default values should have already been set, perhaps by calling
{\tt \packagename\_initialize}.
On exit, individual components of {\tt control} may have been changed
according to the commands found in the specfile. Specfile commands and
the component (see Section~\ref{typecontrol}) of {\tt control}
that each affects are given in Table~\ref{specfile}.

\bctable{|l|l|l|}
\hline
  command & component of {\tt control} & value type \\
\hline
  {\tt error-printout-device} & {\tt \%error} & integer \\
  {\tt printout-device} & {\tt \%out} & integer \\
  {\tt print-level} & {\tt \%print\_level} & integer \\
  {\tt minimum-number-of-iterations} & {\tt \%itmin} & integer \\
  {\tt maximum-number-of-iterations} & {\tt \%itmax} & integer \\
  {\tt maximum-number-of-boundary-iterations} & {\tt \%itmax\_on\_boundary} & integer \\
  {\tt maximum-number-of-inner-iterations} & {\tt \%bitmax} & integer \\
  {\tt number-extra-n-vectors-used} & {\tt \%extra\_vectors} & integer \\
  {\tt relative-accuracy-required} & {\tt \%stop\_relative} & real \\
  {\tt absolute-accuracy-required} & {\tt \%stop\_absolute} & real \\
  {\tt fraction-optimality-required} & {\tt \%fraction\_opt} & real \\
  {\tt stop-as-soon-as-boundary-encountered} & {\tt \%steihaug\_toint} & logical \\
  {\tt space-critical} & {\tt \%space\_critical} & logical \\
  {\tt deallocate-error-fatal} & {\tt \%deallocate\_error\_fatal} & logical \\
\hline

\ectable{\label{specfile}Specfile commands and associated
components of {\tt control}.}

\itt{device} is a scalar \intentin argument of type \integer,
that must be set to the unit number on which the specfile
has been opened. If {\tt device} is not open, {\tt control} will
not be altered and execution will continue, but an error message
will be printed on unit {\tt control\%error}.

\end{description}

%%%%%%%%%%%%%%%%%%%%%% Information printed %%%%%%%%%%%%%%%%%%%%%%%%

\galinfo
If {\tt control\%print\_level} is positive, information about the progress
of the algorithm will be printed on unit {\tt control\-\%\-out}.
If {\tt control\%print\_level = 1}, a single line of output will be produced
for each iteration of the process. So long as the current estimate lies
within the constraint boundary, this will include
the iteration number, the norm of the residual $\bmA\bmx-\bmb$, the
norm of the gradient, and the norm of $\bmx$.
A further message will be printed
if the constraint boundary is encountered during the current iteration.
Thereafter, the one-line summary will also record the value of the Lagrange
multiplier $\lambda$ and the number of Newton steps required to find $\lambda$.
If {\tt control\%print\_level} $\geq$ {\tt 2}, this
output will be increased to provide significant detail of each iteration.
This extra output includes a complete history of the inner iteration required
to solve the ``bi-diagonal'' least-squares subproblem.

%%%%%%%%%%%%%%%%%%%%%% GENERAL INFORMATION %%%%%%%%%%%%%%%%%%%%%%%%

\galgeneral

\galcommon None.
\galworkspace Provided automatically by the module.
\galroutines {\tt \packagename\_solve} calls the
BLAS function {\tt *ROTG}, where {\tt *} is {\tt S} for
the default real version and {\tt D} for the double precision version.
\galmodules {\tt \packagename\_solve} calls the \galahad\ packages
{\tt \libraryname\_SY\-M\-BOLS},
{\tt \libraryname\_SPACE},
{\tt \libraryname\_NORMS},
{\tt \libraryname\_\-ROOTS} and
{\tt GALAHAD\_SPECFILE}.
\galio Output is under control of the arguments
{\tt control\%error}, {\tt control\%out} and {\tt control\%print\_level}.
\galrestrictions $m > 0, \; n  >  0, \;  \Delta  >  0$.
\galportability ISO Fortran~95 + TR 15581 or Fortran~2003.
The package is thread-safe.

%%%%%%%%%%%%%%%%%%%%%% METHOD %%%%%%%%%%%%%%%%%%%%%%%%

\galmethod
The required solution $\bmx$ necessarily satisfies the optimality condition
$\bmA^T ( \bmA \bmx - \bmb ) + \lambda \bmx = 0$, where $\lambda \geq 0$
is a Lagrange
multiplier corresponding to the trust-region constraint
$\|\bmx\|_2  \leq  \Delta$.

\noindent
The method is iterative. Starting  with the vector $\bmu_1 = \bmb$, a
bi-diagonalisation process is used to generate the vectors $\bmv_k$ and
$\bmu_k+1$ so that the $n$ by $k$ matrix $\bmV_k = ( \bmv_1 \ldots \bmv_k)$
and the $m$ by $(k+1)$ matrix $\bmU_k = ( \bmu_1 \ldots \bmu_{k+1})$
together satisfy
\disp{\bmA \bmV_k = \bmU_{k+1} \bmB_k \tim{and}
 \bmb = \|\bmb\| \bmU_{k+1} \bme_1,}
where $\bmB_k$ is $(k+1)$ by $k$ and lower bi-diagonal, $\bmU_k$ and
$\bmV_k$ have orthonormal columns and $\bme_1$ is the first unit vector.
The solution sought is of the form $\bmx_k = \bmV_k \bmy_k$, where $\bmy_k$
solves the bi-diagonal least-squares trust-region problem
\eqn{bls}{ \min \| \bmB_k \bmy - \|\bmb\| \bme_1 \|_2 \tim{subject to}
 \|\bmy\|_2 \leq \Delta.}

If the trust-region constraint is inactive, the solution $\bmy_k$
may be found, albeit indirectly, via the LSQR algorithm of Paige and Saunders
which solves the bi-diagonal least-squares problem
\disp{ \min \| \bmB_k \bmy - \|\bmb\| \bme_1 \|_2}
using a QR factorization of $B_k$. Only the most recent $\bmv_k$ and $\bmu_{k+1}$
are required, and their predecessors discarded, to compute $\bmx_k$ from
$\bmx_{k-1}$. This method has the important property that the iterates
$\bmy$ (and thus $\bmx_k$) generated increase in norm with $k$. Thus
as soon as an LSQR iterate lies outside the trust-region, the required solution
to \req{bls} and thus to the original problem must lie on the boundary of the
trust-region.

If the solution is so constrained, the simplest strategy is to interpolate
the last interior iterate with the newly discovered exterior one to find the
boundary point---the so-called Steihaug-Toint point---between them.
Once the solution is known to lie on the trust-region boundary,
further improvement may be made by solving
\disp{ \min \| \bmB_k \bmy - \|\bmb\| \bme_1 \|_2 \tim{subject to}
\|\bmy\|_2 = \Delta,}
for which the optimality conditions require that $\bmy_k = \bmy(\lambda_k)$
where $\lambda_k$ is the positive root of
\disp{\bmB_k^T ( \bmB_k^{} \bmy(\lambda) - \|\bmb\| \bme_1^{} ) + \lambda
\bmy(\lambda) = 0 \tim{and}  \|\bmy(\lambda)\|_2 = \Delta}
The vector $\bmy(\lambda)$ is equivalently the solution to the
regularized least-squares problem
\disp{ \min  \left \| \vect{ \bmB_k \\ \lambda^{\half} \bmI } y -
 \|\bmb\| \bme_1^{} \right \|}
and may be found efficiently. Given  $\bmy(\lambda)$, Newton's method
is then used to find $\lambda_k$ as the positive root of
$\|\bmy(\lambda)\|_2 = \Delta$. Unfortunately, unlike when the solution
lies in the interior of the trust-region, it is not known how to recur
$\bmx_k$ from $\bmx_{k-1}$ given $\bmy_k$, and a second pass in which
$\bmx_k = \bmV_k \bmy_k$ is regenerated is needed---this need only be done
once $\bmx_k$ has implicitly deemed to be sufficiently close to optimality.
As this second pass is an additional expense, a record is kept of the
optimal objective function values for each value of $k$, and the second
pass is only performed so far as to ensure a given fraction of the
final optimal objective value. Large savings may be made in the second
pass by choosing the required fraction to be significantly smaller than one.

\galreferences
A complete description of the unconstrained case is given by
\vspace*{1mm}

\noindent
C. C. Paige and M. A. Saunders,
LSQR: an algorithm for sparse linear equations and sparse least  squares.
{\em ACM Transactions on Mathematical Software}, 8(1):43--71, 1982

\noindent
and

\noindent
C. C. Paige and M. A. Saunders,
ALGORITHM 583: LSQR: an algorithm for sparse linear equations and
  sparse least squares.
{\em ACM Transactions on Mathematical Software}, 8(2):195--209, 1982.

\noindent
Additional details on how to proceed once the trust-region constraint is
encountered are described in detail in
\vspace*{1mm}

\noindent
C.\ Cartis, N.\ I.\ M.\ Gould and Ph.\ L.\ Toint,
Trust-region and other regularisation of linear
least-squares problems.
{\em BIT} 49(1):21-53 (2009).

%%%%%%%%%%%%%%%%%%%%%% EXAMPLE %%%%%%%%%%%%%%%%%%%%%%%%

\galexample
Suppose we wish to solve a problem in 50 unknowns, whose data is
\disp{\bmA  = \mat{cccc}{
1 & \\ & 1 & \\ & & \ddots & \\ & & & 1 \\
1 & \\ & 2 & \\ & & \ddots & \\ & & & 50
} \;\;
\tim{and}
\bmb  =  \vect{ 1 \\ . \\ . \\ . \\ . \\ . \\ . \\ . \\ 1 },}
with a radius $\Delta = 1$. Suppose in addition that we wish to investigate
the solution beyond the Steihaug-Toint point, but are content with
an approximation which is within 99\% of the best.
Then we may use the following code:

{\tt \small
\VerbatimInput{\packageexample}
}
\noindent
This produces the following output:
{\tt \small
\VerbatimInput{\packageresults}
}


\end{document}
