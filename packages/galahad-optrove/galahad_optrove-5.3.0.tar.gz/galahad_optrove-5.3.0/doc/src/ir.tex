\documentclass{galahad}

% set the package name

\newcommand{\package}{ir}
\newcommand{\packagename}{IR}
\newcommand{\fullpackagename}{\libraryname\_\-\packagename}

\begin{document}

\galheader

%%%%%%%%%%%%%%%%%%%%%% SUMMARY %%%%%%%%%%%%%%%%%%%%%%%%

\galsummary

Given a sparse symmetric matrix $\bmA = \{ a_{ij} \}_{n \times n}$
and the factorization of $\bmA$ found by the \galahad\
package {\tt \libraryname\_SLS}, this package
{\bf solves the system of linear equations $\bmA \bmx = \bmb$ using
iterative refinement.}

%%%%%%%%%%%%%%%%%%%%%% attributes %%%%%%%%%%%%%%%%%%%%%%%%

\galattributes
\galversions{\tt  \fullpackagename\_single, \fullpackagename\_double}.
\galuses {\tt \libraryname\_SY\-M\-BOLS},
{\tt \libraryname\_\-SPACE},
{\tt \libraryname\_\-SMT},
{\tt \libraryname\_\-QPT},
{\tt \libraryname\_SLS},
{\tt \libraryname\_SPECFILE}.
\galdate October 2008.
\galorigin N. I. M. Gould, Rutherford Appleton Laboratory
\gallanguage Fortran~95 + TR 15581 or Fortran~2003.

%%%%%%%%%%%%%%%%%%%%%% HOW TO USE %%%%%%%%%%%%%%%%%%%%%%%%

\galhowto

\subsection{Calling sequences}

The package is available with single, double and (if available) 
quadruple precision reals, and either 32-bit or 64-bit integers. 
Access to the 32-bit integer,
single precision version requires the {\tt USE} statement
\medskip

\hspace{8mm} {\tt USE \fullpackagename\_single}

\medskip
\noindent
with the obvious substitution 
{\tt \fullpackagename\_double},
{\tt \fullpackagename\_quadruple},
{\tt \fullpackagename\_single\_64},
{\tt \fullpackagename\_double\_64} and
{\tt \fullpackagename\_quadruple\_64} for the other variants.


\noindent
If it is required to use more than one of the modules at the same time, 
the derived types
{\tt SMT\_type},
{\tt \packagename\_control\_type},
{\tt \packagename\_inform\_type},
{\tt \packagename\_data\_type},
{\tt SLS\_factors},
(Section~\ref{galtypes})
and the subroutines
{\tt \packagename\_initialize},
{\tt \packagename\_solve}, {\tt \packagename\_terminate}
(Section~\ref{galarguments})
and
{\tt \packagename\_read\_specfile}
(Section~\ref{galfeatures})
must be renamed on one of the {\tt USE} statements.

%%%%%%%%%%%%%%%%%%%%%%%%%%% kinds %%%%%%%%%%%%%%%%%%%%%%%%%%%

%%%%%%%%%%%%%%%%%%%%%% kinds %%%%%%%%%%%%%%%%%%%%%%

\galkinds
We use the terms integer and real to refer to the fortran
keywords {\tt REAL(rp\_)} and {\tt INTEGER(ip\_)}, where 
{\tt rp\_} and {\tt ip\_} are the relevant kind values for the
real and integer types employed by the particular module in use. 
The former are equivalent to
default {\tt REAL} for the single precision versions,
{\tt DOUBLE PRECISION} for the double precision cases
and quadruple-precision if 128-bit reals are available, and 
correspond to {\tt rp\_ = real32}, {\tt rp\_ = real64} and
{\tt rp\_ = real128} respectively as defined by the 
fortran {\tt iso\_fortran\_env} module.
The latter are default (32-bit) and long (64-bit) integers, and
correspond to {\tt ip\_ = int32} and {\tt ip\_ = int64},
respectively, again from the {\tt iso\_fortran\_env} module.


%%%%%%%%%%%%%%%%%%%%%% derived types %%%%%%%%%%%%%%%%%%%%%%%%

\galtypes
Five derived data types are accessible from the package.

%%%%%%%%%%% problem type %%%%%%%%%%%

\subsubsection{The derived data type for holding the matrix}\label{typeprob}
The derived data type {\tt SMT\_type} is used to hold the matrix $\bmA$.
The components of {\tt SMT\_type} are:

\begin{description}

\ittf{n} is a scalar variable of type \integer, that holds
the order $n$ of the matrix  $\bmA$.
\restriction {\tt n} $\geq$ {\tt 1}.

\ittf{ne} is a scalar variable of type \integer, that holds the
number of matrix entries.
\restriction {\tt ne} $\geq$ {\tt 0}.

\ittf{VAL} is a rank-one allocatable array of type \realdp,
and dimension at least {\tt ne}, that holds the values of the entries.
Each pair of off-diagonal entries $a_{ij} = a_{ji} $
is represented as a single entry. Duplicated entries are summed.

\ittf{ROW} is a rank-one allocatable array of type \integer,
and dimension at least {\tt ne}, that holds the row indices of the entries.

\ittf{COL} is a rank-one allocatable array of type \integer,
and dimension at least {\tt ne}, that holds the column indices of the entries.

\end{description}

%%%%%%%%%%% control type %%%%%%%%%%%

\subsubsection{The derived data type for holding control
 parameters}\label{typecontrol}
The derived data type
{\tt \packagename\_control\_type}
is used to hold controlling data. Default values may be obtained by calling
{\tt \packagename\_initialize}
(see Section~\ref{subinit}). The components of
{\tt \packagename\_control\_type}
are:

\begin{description}
\itt{error} is a scalar variable of type \integer, that holds the
stream number for error messages.
Printing of error messages in
{\tt \packagename\_solve} and {\tt \packagename\_terminate}
is suppressed if ${\tt error} \leq {\tt 0}$.
The default is {\tt error = 6}.

\ittf{out} is a scalar variable of type \integer, that holds the
stream number for informational messages.
Printing of informational messages in
{\tt \packagename\_solve} is suppressed if ${\tt out} < {\tt 0}$.
The default is {\tt out = 6}.

\itt{print\_level} is a scalar variable of type \integer,
that is used
to control the amount of informational output which is required. No
informational output will occur if ${\tt print\_level} \leq {\tt 0}$. If
{\tt print\_level = 1} a single line of output will be produced for each
iteration of the process. If {\tt print\_level} $\geq$ {\tt 2} this output
will be increased to provide significant detail of each iteration.
The default is {\tt print\_level = 0}.

\itt{itref\_max} is a scalar variable of type \integer, that holds
the maximum number of iterative refinements which will be allowed.
The default is {\tt itref\_max = 1}.

\itt{acceptable\_residual\_relative}  and {\tt acceptable\_residual\_absolute}
are scalar variables of type \realdp, that
specify an acceptable level for the residual $\bmA \bmx - \bmb$.
In particular, iterative refinement will cease as soon as
$\|\bmA \bmx - \bmb\|_{\infty}$ falls below
$\max (\| \bmb\|_{\infty} \ast$ {\tt acceptable\_residual\_relative},
{\tt acceptable\_residual\_absolute}$)$.
The defaults are {\tt acceptable\_residual\_relative =}{\tt
acceptable\_resi\-dual\_absolute = }$10 u$,
where $u$ is {\tt EPSILON(1.0)} ({\tt EPSILON(1.0D0)} in
{\tt \fullpackagename\_double}).

\itt{required\_residual\_relative} is a scalar variables of type \realdp, that
specify the level for the residual $\bmA \bmx - \bmb$.
In particular, iterative refinement will be deemed to have failed if
$\|\bmA \bmx - \bmb\|_{\infty} >
 \| \bmb\|_{\infty} \ast$ {\tt required\_residual\_relative}.
The defaults is {\tt required\_residual\_relative =} $u^{0.2}$,
where $u$ is {\tt EPSILON(1.0)} ({\tt EPSILON(1.0D0)} in
{\tt \fullpackagename\_double}).

\itt{space\_critical} is a scalar variable of type default \logical, that
may be set \true\ if the user wishes the package to allocate as little
internal storage as possible, and \false\ otherwise. The package may
be more efficient if {\tt space\_critical} is set \false.
The default is {\tt space\_critical = \false}.

\itt{deallocate\_error\_fatal} is a scalar variable of type default \logical,
that may be set \true\ if the user wishes the package to return to the user
in the unlikely event that an internal array deallocation fails,
and \false\ if the package should be allowed to try to continue.
The default is {\tt deallocate\_error\_fatal = \false}.

\itt{prefix} is a scalar variable of type default \character\
and length 30, that may be used to provide a user-selected
character string to preface every line of printed output.
Specifically, each line of output will be prefaced by the string
{\tt prefix(2:LEN(TRIM(prefix))-1)},
thus ignoring the first and last non-null components of the
supplied string. If the user does not want to preface lines by such
a string, they may use the default {\tt prefix = ""}.

\end{description}

%%%%%%%%%%% inform type %%%%%%%%%%%

\subsubsection{The derived data type for holding informational
 parameters}\label{typeinform}
The derived data type
{\tt \packagename\_inform\_type}
is used to hold parameters that give information about the progress and needs
of the algorithm. The components of
{\tt \packagename\_inform\_type}
are:

\begin{description}
\itt{status} is a scalar variable of type \integer, that gives the
current status of the algorithm. See Section~\ref{galerrors} for details.

\itt{alloc\_status} is a scalar variable of type \integer,
that gives the status of the last internal array allocation
or deallocation. This will be 0 if {\tt status = 0}.

\itt{bad\_alloc} is a scalar variable of type default \character\
and length 80, that  gives the name of the last internal array
for which there were allocation or deallocation errors.
This will be the null string if {\tt status = 0}.

\end{description}

%%%%%%%%%%% data type %%%%%%%%%%%

\subsubsection{The derived data type for holding problem data}\label{typedata}
The derived data type
{\tt \packagename\_data\_type}
is used to hold all the data for a particular problem between calls of
{\tt \packagename} procedures.
This data should be preserved, untouched, from the initial call to
{\tt \packagename\_initialize}
to the final call to
{\tt \packagename\_terminate}.

%%%%%%%%%%% factors type %%%%%%%%%%%

\subsubsection{The derived data type for holding factors of a matrix}
\label{typefactors}
The derived data type
{\tt SLS\_FACTORS}
is used to hold the factors and related data for a matrix.
All components are private.

%%%%%%%%%%%%%%%%%%%%%% argument lists %%%%%%%%%%%%%%%%%%%%%%%%

\galarguments
There are three procedures for user calls
(see Section~\ref{galfeatures} for further features):

\begin{enumerate}
\item The subroutine
      {\tt \packagename\_initialize}
      is used to set default values, and initialize private data.
\item The subroutine
      {\tt \packagename\_solve}
      is called to solve $\bmA \bmx = \bmb$; this must have been preceded
      by a call to {\tt SLS\_factorize} to obtain the factors of $\bmA$.
\item The subroutine
      {\tt \packagename\_terminate}
      is provided to allow the user to automatically deallocate array
       components of the private data, allocated by
       {\tt \packagename\_solve},
       at the end of the solution process.
\end{enumerate}
%We use square brackets {\tt [ ]} to indicate \optional\ arguments.

%%%%%% initialization subroutine %%%%%%

\subsubsection{The initialization subroutine}\label{subinit}
 Default values are provided as follows:

\hskip0.5in
{\tt CALL \packagename\_initialize( data, control, inform )}

\begin{description}

\itt{data} is a scalar \intentinout argument of type
{\tt \packagename\_data\_type}
(see Section~\ref{typedata}). It is used to hold data about the problem being
solved.

\itt{control} is a scalar \intentout argument of type
{\tt \packagename\_control\_type}
(see Section~\ref{typecontrol}).
On exit, {\tt control} contains default values for the components as
described in Section~\ref{typecontrol}.
These values should only be changed after calling
{\tt \packagename\_initialize}.

\itt{inform} is a scalar \intentout\ argument of type
{\tt \packagename\_inform\_type}
(see Section~\ref{typeinform}). A successful call to
{\tt \packagename\_initialize}
is indicated when the  component {\tt status} has the value 0.
For other return values of {\tt status}, see Section~\ref{galerrors}.

\end{description}

%%%%%%%%% problem solution subroutine %%%%%%

\subsubsection{The iterative refinement subroutine}
The iterative refinement algorithm is called as follows:

\hskip0.5in
{\tt CALL \packagename\_solve( A, X, data, SLS\_data, control, SLS\_control,  inform, SLS\_inform )}

\begin{description}

\itt{A} is scalar, of \intentin\ and of type {\tt SMT\_TYPE} that holds
the matrix $\bmA$.  All components must be unaltered since the call
to {\tt SLS\_factorize}.

\ittf{X} is an array \intentinout\ argument of dimension {\tt A\%n} and
type \realdp, that must be set on input to contain the vector $\bmb$.
On exit, {\tt X} holds an estimate of the solution $\bmx$

\itt{data} is a scalar \intentinout argument of type
{\tt \packagename\_data\_type}
(see Section~\ref{typedata}). It is used to hold data about the problem being
solved. It must not have been altered {\bf by the user} since the last call to
{\tt \packagename\_initialize}.

\itt{SLS\_data} is scalar, of \intentinout\ and of type {\tt SLS\_data\_type}
that holds the factors of $\bmA$ and related data.
All components must be unaltered since the call to {\tt SLS\_factorize}.

\itt{control} is a scalar \intentin\ argument of type
{\tt \packagename\_control\_type}.
(see Section~\ref{typecontrol}).
Default values may be assigned by calling {\tt \packagename\_initialize}
prior to the first call to {\tt \packagename\_solve}.

\itt{SLS\_control} is a scalar \intentin\ argument of type
{\tt SLS\_control\_type} that is used to control various aspects of the
external packages used to solve the symmetric linear systems that arise.
See the documentation for the \galahad\ package {\tt SLS} for further details.
All components must be unaltered since the call to {\tt SLS\_factorize}.

\itt{inform} is a scalar \intentinout argument of type
{\tt \packagename\_inform\_type}
(see Section~\ref{typeinform}).
A successful call to
{\tt \packagename\_solve}
is indicated when the  component {\tt status} has the value 0.
For other return values of {\tt status}, see Section~\ref{galerrors}.

\itt{SLS\_inform} is a scalar \intentinout argument of type
{\tt SLS\_inform\_type} that is used to pass information
concerning the progress of the external packages used to solve the symmetric
linear systems that arise.
See the documentation for the \galahad\ package {\tt SLS} for further details.

\end{description}

%%%%%%% termination subroutine %%%%%%

\subsubsection{The  termination subroutine}
All previously allocated arrays are deallocated as follows:

\hskip0.5in
{\tt CALL \packagename\_terminate( data, control, inform )}

\begin{description}

\itt{data} is a scalar \intentinout argument of type
{\tt \packagename\_data\_type}
exactly as for
{\tt \packagename\_solve}
that must not have been altered {\bf by the user} since the last call to
{\tt \packagename\_initialize}.
On exit, array components will have been deallocated.

\itt{control} is a scalar \intentin argument of type
{\tt \packagename\_control\_type}
exactly as for
{\tt \packagename\_solve}.

\itt{inform} is a scalar \intentout argument of type
{\tt \packagename\_inform\_type}
exactly as for
{\tt \packagename\_solve}.
Only the component {\tt status} will be set on exit, and a
successful call to
{\tt \packagename\_terminate}
is indicated when this  component {\tt status} has the value 0.
For other return values of {\tt status}, see Section~\ref{galerrors}.

\end{description}

%%%%%%%%%%%%%%%%%%%%%% Warning and error messages %%%%%%%%%%%%%%%%%%%%%%%%

\galerrors
A negative value of  {\tt inform\%status} on exit from
{\tt \packagename\_solve}
or
{\tt \packagename\_terminate}
indicates that an error might have occurred. No further calls should be made
until the error has been corrected. Possible values are:

\begin{description}
\itt{\galerrallocate.} An allocation error occurred. A message indicating
the offending
array is written on unit {\tt control\%error}, and the returned allocation
status and a string containing the name of the offending array
are held in {\tt inform\%alloc\_\-status}
and {\tt inform\%bad\_alloc}, respectively.

\itt{\galerrdeallocate.} A deallocation error occurred.
A message indicating the offending
array is written on unit {\tt control\%error} and the returned allocation
status and a string containing the name of the offending array
are held in {\tt inform\%alloc\_\-status}
and {\tt inform\%bad\_alloc}, respectively.

\itt{\galerrsolve.} Iterative refinement has not reduced the
relative residual by more than
{\tt control\%required\_residual\_relative}.
\end{description}

%%%%%%%%%%%%%%%%%%%%%% Further features %%%%%%%%%%%%%%%%%%%%%%%%

\galfeatures
\noindent In this section, we describe an alternative means of setting
control parameters, that is components of the variable {\tt control} of type
{\tt \packagename\_control\_type}
(see Section~\ref{typecontrol}),
by reading an appropriate data specification file using the
subroutine {\tt \packagename\_read\_specfile}. This facility
is useful as it allows a user to change  {\tt \packagename} control parameters
without editing and recompiling programs that call {\tt \packagename}.

A specification file, or specfile, is a data file containing a number of
"specification commands". Each command occurs on a separate line,
and comprises a "keyword",
which is a string (in a close-to-natural language) used to identify a
control parameter, and
an (optional) "value", which defines the value to be assigned to the given
control parameter. All keywords and values are case insensitive,
keywords may be preceded by one or more blanks but
values must not contain blanks, and
each value must be separated from its keyword by at least one blank.
Values must not contain more than 30 characters, and
each line of the specfile is limited to 80 characters,
including the blanks separating keyword and value.

The portion of the specification file used by
{\tt \packagename\_read\_specfile}
must start
with a "{\tt BEGIN \packagename}" command and end with an
"{\tt END}" command.  The syntax of the specfile is thus defined as follows:
\begin{verbatim}
  ( .. lines ignored by IR_read_specfile .. )
    BEGIN IR
       keyword    value
       .......    .....
       keyword    value
    END
  ( .. lines ignored by IR_read_specfile .. )
\end{verbatim}
where keyword and value are two strings separated by (at least) one blank.
The ``{\tt BEGIN \packagename}'' and ``{\tt END}'' delimiter command lines
may contain additional (trailing) strings so long as such strings are
separated by one or more blanks, so that lines such as
\begin{verbatim}
    BEGIN IR SPECIFICATION
\end{verbatim}
and
\begin{verbatim}
    END IR SPECIFICATION
\end{verbatim}
are acceptable. Furthermore,
between the
``{\tt BEGIN \packagename}'' and ``{\tt END}'' delimiters,
specification commands may occur in any order.  Blank lines and
lines whose first non-blank character is {\tt !} or {\tt *} are ignored.
The content
of a line after a {\tt !} or {\tt *} character is also
ignored (as is the {\tt !} or {\tt *}
character itself). This provides an easy manner to "comment out" some
specification commands, or to comment specific values
of certain control parameters.

The value of a control parameters may be of three different types, namely
integer, logical or real.
Integer and real values may be expressed in any relevant Fortran integer and
floating-point formats (respectively). Permitted values for logical
parameters are "{\tt ON}", "{\tt TRUE}", "{\tt .TRUE.}", "{\tt T}",
"{\tt YES}", "{\tt Y}", or "{\tt OFF}", "{\tt NO}",
"{\tt N}", "{\tt FALSE}", "{\tt .FALSE.}" and "{\tt F}".
Empty values are also allowed for
logical control parameters, and are interpreted as "{\tt TRUE}".

The specification file must be open for
input when {\tt \packagename\_read\_specfile}
is called, and the associated device number
passed to the routine in device (see below).
Note that the corresponding
file is {\tt REWIND}ed, which makes it possible to combine the specifications
for more than one program/routine.  For the same reason, the file is not
closed by {\tt \packagename\_read\_specfile}.

\subsubsection{To read control parameters from a specification file}
\label{readspec}

Control parameters may be read from a file as follows:
\hskip0.5in
\def\baselinestretch{0.8} {\tt \begin{verbatim}
     CALL IR_read_specfile( control, device )
\end{verbatim}}
\def\baselinestretch{1.0}

\begin{description}
\itt{control} is a scalar \intentinout argument of type
{\tt \packagename\_control\_type}
(see Section~\ref{typecontrol}).
Default values should have already been set, perhaps by calling
{\tt \packagename\_initialize}.
On exit, individual components of {\tt control} may have been changed
according to the commands found in the specfile. Specfile commands and
the component (see Section~\ref{typecontrol}) of {\tt control}
that each affects are given in Table~\ref{specfile}.

\bctable{|l|l|l|}
\hline
  command & component of {\tt control} & value type \\
\hline
  {\tt error-printout-device} & {\tt \%error} & integer \\
  {\tt printout-device} & {\tt \%out} & integer \\
  {\tt print-level} & {\tt \%print\_level} & integer \\
  {\tt maximum-refinements} & {\tt \%itref\_max} & integer \\
  {\tt acceptable-residual-relative} &
     {\tt \%acceptable\_residual\_relative} & real \\
  {\tt acceptable-residual-absolute} &
     {\tt \%acceptable\_residual\_absolute} & real \\
  {\tt required-residual-relative} &
     {\tt \%required\_residual\_relative} & real \\
  {\tt space-critical} & {\tt \%space\_critical} & logical \\
  {\tt deallocate-error-fatal} & {\tt \%deallocate\_error\_fatal} & logical \\
% {\tt output-line-prefix} & {\tt \%prefix} & character \\
\hline

\ectable{\label{specfile}Specfile commands and associated
components of {\tt control}.}

\itt{device} is a scalar \intentin argument of type \integer,
that must be set to the unit number on which the specfile
has been opened. If {\tt device} is not open, {\tt control} will
not be altered and execution will continue, but an error message
will be printed on unit {\tt control\%error}.

\end{description}

%%%%%%%%%%%%%%%%%%%%%% Information printed %%%%%%%%%%%%%%%%%%%%%%%%

\galinfo
If {\tt control\%print\_level} is positive, information about the progress
of the algorithm will be printed on unit {\tt control\-\%\-out}.
If {\tt control\%print\_level = 1}, the final value of the norm
of the residual will be given.
If {\tt control\%print\_level > 1}, the norm
of the residual at each iteration will be printed.

%%%%%%%%%%%%%%%%%%%%%% GENERAL INFORMATION %%%%%%%%%%%%%%%%%%%%%%%%

\galgeneral

\galcommon None.
\galworkspace Provided automatically by the module.
\galroutines None.
\galmodules {\tt \packagename\_solve} calls the \galahad\ packages
{\tt \libraryname\_SY\-M\-BOLS},
{\tt \libraryname\_\-SPACE},
{\tt \libraryname\_\-SMT},
{\tt \libraryname\_\-QPT},
{\tt \libraryname\_SLS},
and
{\tt \libraryname\_SPECFILE}.
\galio Output is under control of the arguments
{\tt control\%error}, {\tt control\%out} and {\tt control\%print\_level}.
\galrestrictions None.
\galportability ISO Fortran~95 + TR 15581 or Fortran~2003.
The package is thread-safe.

%%%%%%%%%%%%%%%%%%%%%% METHOD %%%%%%%%%%%%%%%%%%%%%%%%

\galmethod
Iterative refinement proceeds as follows. First obtain the floating-point
solution to $\bmA \bmx = \bmb$ using the factors of $\bmA$. Then iterate
until either the desired residual accuracy (or the iteration limit is
reached) as follows: evaluate the residual $\bmr = \bmb - \bmA \bmx$,
find the floating-point solution $\delta \bmx$ to $\bmA \delta \bmx = \bmr$,
and replace $\bmx$ by $\bmx + \delta \bmx$.

%%%%%%%%%%%%%%%%%%%%%% EXAMPLE %%%%%%%%%%%%%%%%%%%%%%%%

\galexample
Suppose we wish to solve the set of equations
\disp{\mat{ccccc}{ 2 & 3 &   &   &   \\
                   3 &   & 4 &   & 6 \\
                     & 4 & 1 & 5 &   \\
                     &   & 5 &   &   \\
                     & 6 & &     & 1 } \bmx =
       \vect{ 8 \\ 45 \\ 31 \\ 15 \\ 17}}
Then we may use the following code

{\tt \small
\VerbatimInput{\packageexample}
}
\noindent
with the following data
{\tt \small
\VerbatimInput{\packagedata}
}
\noindent
This produces the following output:
{\tt \small
\VerbatimInput{\packageresults}
}


\end{document}
