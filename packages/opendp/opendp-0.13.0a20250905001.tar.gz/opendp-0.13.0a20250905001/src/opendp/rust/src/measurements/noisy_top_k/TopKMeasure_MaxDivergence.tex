\documentclass{article}
% common styling and macros shared by all proof files

\usepackage[top=1in, right=1in, left=1in, bottom=1.5in]{geometry}

\usepackage{amsmath,amsthm,amsfonts,amssymb,amscd}
\usepackage{listings}
\usepackage{hyperref}
\usepackage{xcolor}
\usepackage{xr}

\usepackage{enumerate} 
\usepackage{physics}
\usepackage{fancyhdr}
\usepackage{hyperref}
\usepackage{graphicx}
\usepackage{tcolorbox}
\usepackage{catchfile}
\usepackage{pdftexcmds}
\usepackage[T1]{fontenc}

% hyperref
\hypersetup{
  colorlinks=true,
  linkcolor=blue,
  linkbordercolor={0 0 1}
}

% \contrib macro to indicate inclusion in "contrib".
\usepackage{tcolorbox}
\newtcolorbox{warn_box}{colback=red!5!white,colframe=red!75!black}
\newcommand{\contrib}{{\begin{warn_box}This proof resides in \textbf{``contrib''} because it has not completed the vetting process.\end{warn_box}}} 
\newcommand{\floatingPoint}{{\begin{warn_box}This implementation is susceptible to floating-point vulnerabilities.\end{warn_box}}} 

% asOfCommit macro to version a code dependency. Arguments:
%    #1: relative path to file you are dependent on
%    #2: commit hash it was last edited. If outdated, this should be the second hash in the footnoote. Otherwise,
%            git log -n 1 --pretty=format:%h -- path/to/file.rs
\makeatletter
\ifnum\pdf@shellescape=1
   % "private" command that builds a link to a blob
  \newcommand{\linkOpendpBlob}[3]{%
    \href{https://github.com/opendp/opendp/blob/#1/#2#3}{\path{#3} at commit #1}}

  % latex macro expansion has a separate phase for \input evaluation
  %     immediately evaluate a command to write a temp file to ./out containing the current directory
  \immediate\write18{[ ! -f out/cwd.txt ] && (mkdir -p out && git rev-parse --show-prefix | sed "s|_|\@backslashchar\@backslashchar\@backslashchar_|g" > out/cwd.txt)}
  %     ...and then retrieve the current working directory by loading the temp file
  \CatchFileDef\GitWorkingDir{out/cwd.txt}{\endlinechar=-1}

  % command for building the (up to date) or (outdated) status
  \newcommand{\fileStatus}[2]{%
  \setbox0=\hbox{\input|"git --no-pager log -n1 --pretty='\@percentchar H' #1 | grep -E '^#2.*'"\unskip}\ifdim\wd0=0pt
        (outdated\footnote{See new changes with \texttt{git diff #2..\input|"git --no-pager log -n1 --pretty='\@percentchar h' #1" \GitWorkingDir\path{#1}}})\else
        (up to date)\fi
  }

  \newcommand{\asOfCommit}[2]{%
      % permalink the target
      \linkOpendpBlob{#2}{\GitWorkingDir}{#1}
      % conditionally add (outdated) or (up to date) depending on matching commit hash
      \fileStatus{#1}{#2}%
  }
\else
  % simplified command if shell-escape not enabled
  \newcommand{\asOfCommit}[2]{#1 at commit #2 (unknown status\footnote{Shell-escape is not enabled. Enable \texttt{--shell-escape} to check if this proof is up-to-date with the code.})}
\fi
\makeatother

% \vettingPR macro to link a PR. Arguments:
%    #1: PR number
\newcommand{\vettingPR}[1]{\href{https://github.com/opendp/opendp/pull/#1}{Pull Request \##1}}

% for links to rustdoc items in OpenDP. Arguments:
%    #1: path to item, and designation as trait, struct, fn, etc.
%    #2: item name
\makeatletter
\ifnum\pdf@shellescape=1
  % latex macro expansion has a separate phase for \input evaluation
  %     immediately evaluate a command to write a temp file to ./out containing the base path
  \immediate\write18{[ ! -f out/rustdoc.txt ] && mkdir -p out && ([ -z `kpsewhich --var-value OPENDP_RUSTDOC_PORT` ] && echo "https://docs.rs/opendp/`head -n 1 \@backslashchar`git rev-parse --show-toplevel\@backslashchar`/VERSION | sed 's|.*-dev.*|latest|g'`" || echo "http://localhost:`kpsewhich --var-value OPENDP_RUSTDOC_PORT`") > out/rustdoc.txt}
  %     ...and then retrieve the base path by loading the temp file
  \CatchFileDef\OpenDPRustdocBase{out/rustdoc.txt}{\endlinechar=-1}
\else
  % if shell commands are not enabled, just claim latest
  \newcommand{\OpenDPRustdocBase}{https://docs.rs/opendp/latest}
\fi
\makeatother
\newcommand{\rustdoc}[2]{\href{\OpenDPRustdocBase/opendp/#1.#2.html}{\texttt{#2}}}

% for links to external dependencies. Arguments:
%    #1: crate name
%    #2: path to item, and designation as trait, struct, fn, etc.
%    #3: item name
\newcommand{\docsrs}[3]{\href{https://docs.rs/#1/latest/#1/#2.#3.html}{\texttt{#3}}}

% minted (pseudocode)
\definecolor{codegreen}{rgb}{0,0.6,0}
\definecolor{codegray}{rgb}{0.5,0.5,0.5}
\definecolor{codepurple}{rgb}{0.58,0,0.82}
\definecolor{backcolour}{rgb}{0.95,0.95,0.92}

\lstdefinestyle{mystyle}{
    backgroundcolor=\color{backcolour},   
    commentstyle=\color{codegreen},
    keywordstyle=\color{magenta},
    numberstyle=\tiny\color{codegray},
    stringstyle=\color{codepurple},
    basicstyle=\ttfamily\footnotesize,
    breakatwhitespace=false,         
    breaklines=true,                 
    captionpos=b,                    
    keepspaces=true,                 
    numbers=left,                    
    numbersep=5pt,                  
    showspaces=false,                
    showstringspaces=false,
    showtabs=false,                  
    tabsize=2
}

\lstset{style=mystyle}

% common commands
\theoremstyle{definition}
\newtheorem{theorem}{Theorem}[section]
\newtheorem{lemma}[theorem]{Lemma}
\newtheorem{definition}[theorem]{Definition}
\newtheorem{warning}{Warning}
\newtheorem{corollary}{Corollary}
\newtheorem{proposition}{Proposition}
\newtheorem{remark}{Remark}
\newtheorem{observation}{Observation}
\newtheorem{note}{Note}

\newcommand{\vicki}[1]{{ {\color{olive}{(vicki)~#1}}}}
\newcommand{\hanwen}[1]{{ {\color{purple}{(hanwen)~#1}}}}
\newcommand{\zach}[1]{{ {\color{red}{(zach)~#1}}}}

\newcommand{\MultiSet}{\mathrm{MultiSet}}
\newcommand{\len}{\mathrm{len}}
\newcommand{\din}{\texttt{d\_in}}
\newcommand{\dout}{\texttt{d\_out}}
\newcommand{\T}{\texttt{T} }
\newcommand{\F}{\texttt{F} }
\newcommand{\Map}{\texttt{Map}}
\newcommand{\X}{\mathcal{X}}
\newcommand{\Y}{\mathcal{Y}}
\newcommand{\True}{\texttt{True}}
\newcommand{\False}{\texttt{False}}
\newcommand{\clamp}{\texttt{clamp}}
\newcommand{\function}{\texttt{function}}
\newcommand{\float}{\texttt{float }}
\newcommand{\questionc}[1]{\textcolor{red}{\textbf{Question:} #1}}


\newcommand{\validTransformation}[2]{%
  \begin{theorem}
  For every setting of the input parameters #1 to #2 such that the given preconditions
  hold, #2 raises an error (at compile time or run time) or returns a valid transformation. A valid transformation has the following properties:
  \begin{enumerate}
      \item \textup{(Data-independent runtime errors).}
      For every pair of members $x$ and $x'$ in \texttt{input\_domain},
      $\texttt{invoke}(x)$ and $\texttt{invoke}(x')$ either both return the same error or neither return an error. 

      \item \textup{(Appropriate output domain).} 
      For every member $x$ in \texttt{input\_domain}, $\function(x)$ is in \texttt{output\_domain} or raises a data-independent runtime error.
      
      \item \textup{(Stability guarantee).} 
      For every pair of members $x$ and $x'$ in \texttt{input\_domain} and for every pair $(\din, \dout)$, 
      where \din\ has the associated type for \texttt{input\_metric} and \dout\ has the associated type for \\ \texttt{output\_metric}, 
      if $x, x'$ are \din-close under \texttt{input\_metric}, $\texttt{stability\_map}(\din)$ does not raise an error,
      and $\texttt{stability\_map}(\din) = \dout$, 
      then $\function(x), \function(x')$ are $\dout$-close under \texttt{output\_metric}.
  \end{enumerate}
  \end{theorem}
}


\newcommand{\validMeasurement}[2]{%
  For every setting of the input parameters #1 to #2 such that the given preconditions
  hold, #2 raises an error (at compile time or run time) or returns a valid measurement. A valid measurement has the following properties:
  \begin{enumerate}
      \item \textup{(Data-independent runtime errors).}
      For every pair of members $x$ and $x'$ in \texttt{input\_domain},
      $\texttt{invoke}(x)$ and $\texttt{invoke}(x')$ either both return the same error or neither return an error. 

      \item \textup{(Privacy guarantee).}
      For every pair of members $x$ and $x'$ in \texttt{input\_domain} and for every pair $(\din, \dout)$,
      where \din\ has the associated type for \texttt{input\_metric} and \dout\ has the associated type for \\ \texttt{output\_measure},
      if $x, x'$ are \din-close under \texttt{input\_metric}, $\texttt{privacy\_map}(\din)$ does not raise an error,
      and $\texttt{privacy\_map}(\din) = \dout$,
      then $\function(x), \function(x')$ are $\dout$-close under \texttt{output\_measure}.
  \end{enumerate}
}

\newcommand{\validPostprocessor}[2]{%
  For every setting of the input parameters #1 to #2 such that the given preconditions
  hold, #2 raises an error (at compile time or run time) or returns a valid postprocessor. A valid postprocessor has the following property:
  \begin{enumerate}
      \item \textup{(Data-independent errors).}
      For every pair of members $x$ and $x'$ in \texttt{input\_domain},
      $\function(x), \function(x')$ either both raise the same error, 
      or neither raise an error.
  \end{enumerate}
}

\newcommand{\validOdometer}[2]{%
  For every setting of the input parameters #1 to #2 such that the given preconditions
  hold, #2 raises an error (at compile time or run time) or returns a valid odometer. 
  A valid odometer has the following properties:
  \begin{enumerate}
      \item \textup{(Data-independent runtime errors).}
      For every pair of members $x$ and $x'$ in \texttt{input\_domain},
      $\texttt{invoke}(x)$ and $\texttt{invoke}(x')$ either both return the same error or neither return an error. 

      \item \textup{(Valid odometer queryable).}
      For every member $x$ in \texttt{input\_domain},
      where $\function(x)$ does not raise an error,
      $\function(x)$ returns a valid odometer queryable.
  \end{enumerate}
}

\newcommand{\validOdometerQueryable}[2]{%
  For every setting of the input parameters #1 to #2 such that the given preconditions
  hold, #2 raises a data-independent error or returns a valid odometer queryable (\texttt{queryable}). 
  A valid odometer queryable is an \rustdoc{core/type}{OdometerQueryable} with the following properties:
  \begin{enumerate}
    \item \textup{(Data-independent errors).}
    For every pair of members $x$ and $x'$ in \texttt{input\_domain}, and every query $q$,
    $\texttt{queryable.invoke}_x(q)$ and $\texttt{queryable.invoke}_{x'}(q)$ either both return the same error or neither return an error. 

    \item \textup{(Privacy Guarantee).}
    For every pair of members $x$ and $x'$ in \texttt{input\_domain}, 
    every adversary $\mathcal{A}$, and for every pair $(\din, \dout)$, 
    where \din\ has the associated type for \texttt{input\_metric} and \dout\ has the associated type for \texttt{output\_measure}, 
    if $x$ and $x'$ are \din-close under \texttt{input\_metric}, $\texttt{queryable.eval(privacy\_loss)}$ does not return an error, 
    and $\texttt{queryable.eval(privacy\_loss)} = \dout$, 
    then $\texttt{View}(\mathcal{A} \leftrightarrow \texttt{queryable}_x), \texttt{View}(\mathcal{A} \leftrightarrow \texttt{queryable}_{x'})$ are $\dout$-close under \texttt{output\_measure},
    where $\texttt{queryable}_x$ denotes all messages received by $\mathcal{A}$ under $x$.
  \end{enumerate}
}

\allowdisplaybreaks

\title{\texttt{impl TopKMeasure for MaxDivergence}}
\author{Tudor Cebere \and Michael Shoemate}
\begin{document}
\maketitle
\contrib

This document proves soundness of \rustdoc{measurements/noisy_top_k/exponential}{permute\_and\_flip} \cite{mckenna2020permute} in \asOfCommit{mod.rs}{e62b0aa2}.
\texttt{permute\_and\_flip} noisily selects the index of the greatest score from a vector of input scores.

Permute and flip is equivalent to report noisy max with exponential noise \cite{ding2021permute}.
Report noisy max exponential is implemented via permute and flip because of its discrete nature.
Implementation-wise, we will follow permute-and-flip,
yet prove the correctness of the algorithm via this equivalence.

\section{Hoare Triple}
\subsection*{Precondition}
\subsubsection*{Compiler-verified}
\begin{itemize}
    \item Method \texttt{noisy\_top\_k}
        \textit{Types consistent with pseudocode.}
    \item Method \texttt{privacy\_map}
        \textit{Types consistent with pseudocode.}
\end{itemize}

\subsubsection*{Caller-verified}
\begin{itemize}
    \item Method \texttt{noisy\_top\_k}
        \begin{itemize}
            \item \texttt{x} elements are non-null.
            \item \texttt{scale} is finite and non-negative.
        \end{itemize}
    \item Method \texttt{privacy\_map}
        \begin{itemize}
            \item \texttt{d\_in} is non-null and positive.
            \item \texttt{scale} is non-null and positive.
        \end{itemize}
\end{itemize}

\subsection*{Pseudocode}
\label{sec:python-pseudocode}
\lstinputlisting[language=Python,firstline=2,escapechar=|]{./pseudocode/TopKMeasure_MaxDivergence.py}

\subsection*{Postcondition}
\begin{theorem}
    The implementation is consistent with all associated items in the \rustdoc{measurements/noisy\_top\_k/trait}{TopKMeasure} trait.
    \begin{enumerate}
        \item Method \texttt{noisy\_top\_k}:
        \begin{itemize}
            \item Returns the index of the top element $z_i$,
            where each $z_i \sim \mathrm{DISTRIBUTION}(\mathrm{shift}=y_i, \mathrm{scale}=\texttt{scale})$,
            and each $y_i = -x_i$ if \texttt{negate}, else $y_i = x_i$,
            $k$ times with removal.
            \item Errors are data-independent, except for exhaustion of entropy.
        \end{itemize}
        \item Method \texttt{privacy\_map}:
        For any $x, x'$ where $d_\mathrm{in} \ge d_\mathrm{Range}(x, x')$,
        return $d_\mathrm{out} \ge D_\mathrm{self}(f(x), f(x'))$,
        where $f(x) = \mathrm{noisy\_top\_k}(x=x, k=1, \mathrm{scale}=\mathrm{scale})$.
    \end{enumerate}
\end{theorem}


\begin{definition}
    \label{def:Exponential}
    A random variable follows the Exponential distribution if it has density
    \begin{equation}
        f(x) = \frac{1}{\beta} e^{-z}
    \end{equation}
    where $z = \frac{x - \mu}{\beta}$,
    $\mu$ is the shift (location) parameter and $\beta$ is the scale parameter.
\end{definition}

\begin{proof}[Proof of postcondition: \texttt{noisy\_top\_k}]
    The preconditions of \rustdoc{measurements/noisy_top_k/exponential/fn}{exponential\_noisy\_max} are met,
    therefore by the postcondition of \texttt{exponential\_top\_k},
    the postcondition of \texttt{noisy\_top\_k} is satisfied.
\end{proof}


Before proving the privacy guarantees, we state a few required definitions and lemmas:
\begin{definition}
    \label{definition:rnm-exp}
    Report noisy max with exponential noise computes the index of the maximum element from a set of candidates $u \in \din $,
    adds isotropic exponential noise $Z_i \sim \mathrm{Exp}(1/\lambda)$ to each element in the candidate set $u$ and returns the maximum index as follows:
    \begin{equation}
        \texttt{RNM-Exp}(s) = \mathrm{argmax}_i(s_i + Z_i), Z_i \sim \mathrm{Exp}(\lambda)
    \end{equation}
\end{definition}

\begin{lemma}
    \label{lemma:equivalence_of_rnme}
    The permute-and-flip mechanism is equivalent to the report-noisy-max with exponential noise mechanism.
\end{lemma}
See \cite{ding2021permute} for proof of Lemma~\ref{lemma:equivalence_of_rnme}.


\begin{lemma}
    \label{lemma:diff_cdf}
    Let $X_1, X_2 \sim \mathrm{Exp}(\lambda), \Delta \geq 0$, then
    \begin{equation}
        \Pr[X_1 - X_2 \geq \Delta] = e^{-\Delta\lambda} \Pr[X_1 - X_2 \geq 0]
    \end{equation}
\end{lemma}

\begin{proof}[Proof of Lemma~\ref{lemma:diff_cdf}]
    \begin{align}
    & \Pr[X_1 - X_2 \geq \Delta] \\
    & = 1 - \Pr[X_1 \leq \Delta + X_2] \\
    & = 1 - \int^\infty_0 \Pr[X_1 \leq \Delta + X_2 | X_2 = x] \Pr[X_2 = x] dx && \text{by Law of Total Probability} \\
    & = 1 - \int^\infty_0 \Pr[X_1 \leq \Delta + x] \Pr[X_2 = x] dx  && \text{by the fact that } \Delta > 0 \\
    & = 1 - \int^\infty_0 \lambda(1 - e^{-(x + \Delta)\lambda})e^{-x\lambda}dx  \\
    & = 1 - \lambda \int^\infty_0 e^{-x\lambda} dx + \lambda e^{-\Delta\lambda} \int_0^\infty e^{-2x\lambda}dx \\
    & = 1 - 1 + e^{-\Delta\lambda} / 2 && \text{$\Pr[X_1 - X_2 \leq 0] = \Pr[X_1 - X_2 \geq 0] = 1/2$} \\
    & = e ^{-\Delta\lambda} \Pr[X_1 - X_2 \geq 0]
    \end{align}
\end{proof}

\newcommand\logeq{\mathrel{\vcentcolon\Leftrightarrow}}

\begin{lemma}
    \label{lemma:equiv}
    Let $u, v \in \texttt{input\_domain}$ be two vectors of scores.
    Assume $u$, $v$ in \texttt{input\_domain} are \din-close under \rustdoc{metrics/struct}{LInfDistance} and $\texttt{privacy\_map}(\din) \le \dout$.
    Let $Z^* = min_{Z_i} \{ u_i + Z_i \geq u_j + Z_j \}, \forall i \neq j$. Then
    \begin{equation}
        \ln \left(\frac{\Pr[\texttt{function}(u) = i]}{\Pr[\texttt{function}(v) = i]}\right) =
        \ln \left(\frac{\Pr[Z_i \geq  Z^*]}{\Pr[Z_i \geq  Z^* + \din]} \right) \\
    \end{equation}

    \begin{proof}
    \begin{align}
        & \ln \left( \frac{{\Pr[\texttt{function}(u) = i]}}{\Pr[\texttt{function}(v) = i]} \right)  && \\
        & = \ln \left( \frac{\Pr[\texttt{RNM-Exp(u)} = i]}{\Pr[\texttt{RNM-Exp}(v) = i]} \right) && \text{by Lemma \ref{lemma:equivalence_of_rnme}} \\
        & = \ln \left( \frac{\Pr[\text{argmax}_k(u_k + Z_k) = i]}{\Pr[\text{argmax}_k(v_k + Z_k) = i]} \right) && \text{by Definition \ref{definition:rnm-exp}} \\
 \intertext{Observe that for a fixed $i$, report noisy max outputs $i$ if:}
        & u_i + Z^* \geq u_j + Z_j, \forall i \neq j & \iff && \\
        & u_i + (v_i - v_i) + Z^* \geq u_j + (v_j - v_j) + Z_j & \iff && \\
        & v_i + (u_i - v_i) + Z^* \geq v_j + (u_j - v_j) + Z_j & \iff && \\
        & v_i + ((u_i - v_i) -(u_j - v_j) + Z^*) \geq v_j + Z_j & \iff && \\
        & v_i + (\Delta + Z^*) \geq v_j + Z_j
 \intertext{In other words, if $Z_i \geq (\Delta + Z^*)$, then $\texttt{function}(u) = \texttt{function}(v) = i$. This yields us:}
        & \ln \left( \frac{\Pr[\text{argmax}_k(u_k + Z_k) = i]}{\Pr[\text{argmax}_k(v_k + Z_k) = i]} \right) = \ln \left( \frac{\Pr[Z_i \geq Z^*]}{\Pr[Z_i \geq \Delta + Z^*]} \right)
    \end{align}
\end{proof}
\end{lemma}

\begin{proof}[Proof of postcondition: \texttt{privacy\_map}]
\begin{align}
    & \max_{u \sim v} D_\infty(M(u) | M(v)) \\
    & = \max_{u \sim v} max_i \ln\left(\frac{\Pr[function(u) = i]}{\Pr[function(v) = i]}\right) && \\
    & = \max_{u \sim v} max_i \ln\left(\frac{\Pr[\texttt{RNM-Exp(u)} = i]}{\Pr[\texttt{RNM-Exp}(v) = i]} \right) &&  \text{by Lemma \ref{lemma:equivalence_of_rnme}}\\
    & = \max_{u \sim v} max_i \ln\left(\frac{\Pr[\text{argmax}_k(u_k + Z_k) = i]}{\Pr[\text{argmax}_k(v_k + Z_k) = i]}\right) && \text{by Definition \ref{definition:rnm-exp}}  \\
    & = \max_{u \sim v} max_i \ln\left(\frac{\Pr[Z_i \geq Z^*]}{\Pr[Z_i \geq Z^* + \din]}\right) && \text{by Lemma \ref{lemma:equiv}}\\
    & \le \frac{\din}{\texttt{scale}} && \text{by Lemma \ref{lemma:diff_cdf}}
\end{align}
\end{proof}

\bibliographystyle{plain}
\bibliography{references.bib}

\end{document}
