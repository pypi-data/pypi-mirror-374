
\documentclass[10pt]{article}
%%%%%%%%%%%%%%%%%%%%%%%%%%%%%%%%%%%%%%%%%%%%%%%%%%%%%%%%%%%%%%%%%%%%%%%%%%%%%%%%%%%%%%%%%%%%%%%%%%%%%%%%%%%%%%%%%%%%%%%%%%%%%%%%%%%%%%%%%%%%%%%%%%%%%%%%%%%%%%%%%%%%%%%%%%%%%%%%%%%%%%%%%%%%%%%%%%%%%%%%%%%%%%%%%%%%%%%%%%%%%%%%%%%%%%%%%%%%%%%%%%%%%%%%%%%%
\usepackage{amsfonts}
\usepackage{amssymb}
\usepackage{sw20elba}

%TCIDATA{OutputFilter=LATEX.DLL}
%TCIDATA{Version=5.50.0.2890}
%TCIDATA{<META NAME="SaveForMode" CONTENT="1">}
%TCIDATA{BibliographyScheme=Manual}
%TCIDATA{Created=Sunday, July 28, 2013 08:48:01}
%TCIDATA{LastRevised=Wednesday, August 21, 2013 07:07:37}
%TCIDATA{<META NAME="GraphicsSave" CONTENT="32">}
%TCIDATA{<META NAME="DocumentShell" CONTENT="Articles\SW\mrvl">}
%TCIDATA{CSTFile=LaTeX article (bright).cst}

\newtheorem{theorem}{Theorem}
\newtheorem{axiom}[theorem]{Axiom}
\newtheorem{claim}[theorem]{Claim}
\newtheorem{conjecture}[theorem]{Conjecture}
\newtheorem{corollary}[theorem]{Corollary}
\newtheorem{definition}[theorem]{Definition}
\newtheorem{example}[theorem]{Example}
\newtheorem{exercise}[theorem]{Exercise}
\newtheorem{lemma}[theorem]{Lemma}
\newtheorem{notation}[theorem]{Notation}
\newtheorem{problem}[theorem]{Problem}
\newtheorem{proposition}[theorem]{Proposition}
\newtheorem{remark}[theorem]{Remark}
\newtheorem{solution}[theorem]{Solution}
\newtheorem{summary}[theorem]{Summary}
\newenvironment{proof}[1][Proof]{\noindent\textbf{#1.} }{{\hfill $\Box$ \\}}
\input{tcilatex}
\addtolength{\textheight}{30pt}
\addtolength{\textwidth}{140pt}
\addtolength{\oddsidemargin}{-1in}
\addtolength{\evensidemargin}{-1in}

\begin{document}

\title{Algebra 4.1}
\author{Michael Vaughan-Lee}
\date{July 2013}
\maketitle

The number of immediate descendants of algebra 4.1 of order $p^{7}$ is 1361
if $p=3$. For $p>3$ it is $p^{5}+2p^{4}+7p^{3}+25p^{2}+88p+270+(p+4)\gcd
(p-1,3)+\gcd (p-1,4)$.

If $L$ is an immediate descendant of 4.1 of order $p^7$ then $L$ is
generated by $a,b,c,d$, $L_2$ has order $p^3$, and $L_3=\{0\}$.

\section{$L$ abelian}

\[
\langle a,b,c,d\,|\,ba,ca,da,cb,db,dc,pd,\,\text{class }2\rangle . 
\]

\section{$L^{2}$ has order $p$}

If $L^{2}$ has order $p$ then we can assume that $L^{2}$ is generated by $ba$
and that one of the following two sets of commutator relations hold: 
\begin{eqnarray*}
ca &=&da=cb=db=dc=0, \\
ca &=&da=cb=db=0,\,dc=ba.
\end{eqnarray*}%
There are 7 algebras in the first case, and 4 in the second case.

\section{$L^{2}$ has order $p^{2}$}

If $L^{2}$ has order $p^{2}$ then we can assume that one of the following
sets of commutator relations holds: 
\begin{eqnarray*}
da &=&cb=db=dc=0, \\
ca &=&da=cb=db=0, \\
da &=&cb=dc=0,\,db=ca, \\
da &=&cb=0,\,db=ca,\,dc=\omega ba.
\end{eqnarray*}%
Note that $L^{2}$ is generated by $ba,ca$ in all but the second of these
algebras. In the second algebra, $L^{2}$ is generated by $ba,dc$. We obtain $%
2p+29$ algebras in the first case, $(p^{2}-1)/2+4p+30$ in the second, $3p+26$
in the third, and $(p^{2}-1)/2+2p+6$ in the fourth.

In solving the isomorphism problem in Case 4, we have the following
presentation:%
\[
\langle a,b,c,d\,|\,da,cb,db-ca,dc-\omega ba,pa,pb-xba-yca,pc-zba-tca,\,%
\text{class }2\rangle , 
\]%
where $\left( 
\begin{array}{ll}
x & y \\ 
z & t%
\end{array}%
\right) $ runs over a set of representatives for the equivalence classes of
non-singular matrices $A$ under the equivalence relation given by%
\[
A\sim \alpha ^{-1}\left( 
\begin{array}{ll}
\mu & \nu \\ 
\pm \omega \nu & \pm \mu%
\end{array}%
\right) A\left( 
\begin{array}{ll}
\mu & \nu \\ 
\pm \omega \nu & \pm \mu%
\end{array}%
\right) ^{-1}. 
\]%
There are $(p+1)^{2}/2$ equivalence classes.

There is a \textsc{Magma} program in notes4.1case4.m to compute a set of
representative matrices $A$.

\section{$L^{2}$ has order $p^{3}$}

If $L^2$ has order $p^3$ then $L$ must have the same commutator structure as
one of 7.15 -- 7.20 from the list of nilpotent Lie algebras of dimension 7
over $\mathbb{Z}_p$, so we can assume that one of the following sets of
commutator relations holds: 
\begin{eqnarray*}
da &=&db=dc=0, \\
ca &=&da=db=0, \\
ca &=&da=dc=0, \\
ca &=&da=0,\,dc=ba, \\
da &=&0,\,db=ca,\,dc=cb, \\
da &=&0,\,db=\omega ca,\,dc=ba.
\end{eqnarray*}

In Case 1 we have $3p+18$ algebras.

In Case 2 we have $\allowbreak \frac{77}{2}p+\frac{173}{2}+11p^{2}+\frac{5}{2%
}p^{3}+\frac{1}{2}p^{4}$ algebras, but you need to add 2 if $p=1\func{mod}3$.

In Case 3 we have $\allowbreak p^{2}+3p+15$, but again you need to add 2 if $%
p=1\func{mod}3$.

In Case 4 we have $3p^{2}+13p+31$ algebras, but we need to add 2 if $p=1%
\func{mod}4$ and add 2 if $p=1\func{mod}3$.

In Case 5 we have 550 algebras when $p=3$ and 
\begin{eqnarray*}
\allowbreak p^{5}+p^{4}+4p^{3}+6p^{2}+18p+19\text{ if }p &=&1\func{mod}3, \\
\allowbreak p^{5}+p^{4}+4p^{3}+6p^{2}+16p+17\text{ if }p &=&2\func{mod}3.
\end{eqnarray*}

In Case 6 we have $\frac{9}{2}p+\frac{13}{2}+3p^{2}+\frac{1}{2}p^{4}+\frac{1%
}{2}p^{3}$ algebras.

We need computer programs to sort out the isomorphism problem in Case 5 and
in Case 6.

\subsection{Case 5}

Let $L$ satisfy $da=0,\,db=ca,\,dc=cb$. It is convenient to replace $b$ by $%
b+d$, so that $L$ satisfies $da=cb=0$, $db=ca$. So $L^{2}$ is generated by $%
ba$, $ca$ and $dc$, and $pL\leq L^{2}$. It is fairly easy to see that if $%
a^{\prime },b^{\prime },c^{\prime },d^{\prime }$ generate $L$ and satisfy $%
d^{\prime }a^{\prime }=c^{\prime }b^{\prime }=0$, $d^{\prime }b^{\prime
}=c^{\prime }a^{\prime }$, then (modulo $L^{2}$) 
\begin{eqnarray*}
a^{\prime } &=&\alpha \lambda a+\beta \lambda b+\beta \mu c-\alpha \mu d, \\
b^{\prime } &=&\gamma \lambda a+\delta \lambda b+\delta \mu c-\gamma \mu d,
\\
c^{\prime } &=&\gamma \nu a+\delta \nu b+\delta \xi c-\gamma \xi d, \\
d^{\prime } &=&-\alpha \nu a-\beta \nu b-\beta \xi c+\alpha \xi d
\end{eqnarray*}%
with $(\alpha ,\beta )$ and $(\gamma ,\delta )$ linearly independant, and
with $(\lambda ,\mu )$ and $(\nu ,\xi )$ linearly independant. Furthermore 
\[
\left( 
\begin{array}{c}
b^{\prime }a^{\prime } \\ 
c^{\prime }a^{\prime } \\ 
d^{\prime }c^{\prime }%
\end{array}%
\right) =(\alpha \delta -\beta \gamma )\left( 
\begin{array}{ccc}
\lambda ^{2} & 2\lambda \mu & \mu ^{2} \\ 
\lambda \nu & \lambda \xi +\mu \nu & \mu \xi \\ 
\nu ^{2} & 2\nu \xi & \xi ^{2}%
\end{array}%
\right) \left( 
\begin{array}{c}
ba \\ 
ca \\ 
dc%
\end{array}%
\right) . 
\]%
So we consider orbits of $4\times 3$ matrices $A$ (representing $pa,pb,pc,pd$%
) under transformations of the form 
\[
A\longmapsto (\alpha \delta -\beta \gamma )^{-1}\left( 
\begin{array}{cccc}
\alpha \lambda & \beta \lambda & \beta \mu & -\alpha \mu \\ 
\gamma \lambda & \delta \lambda & \delta \mu & -\gamma \mu \\ 
\gamma \nu & \delta \nu & \delta \xi & -\gamma \xi \\ 
-\alpha \nu & -\beta \nu & -\beta \xi & \alpha \xi%
\end{array}%
\right) A\left( 
\begin{array}{ccc}
\lambda ^{2} & 2\lambda \mu & \mu ^{2} \\ 
\lambda \nu & \lambda \xi +\mu \nu & \mu \xi \\ 
\nu ^{2} & 2\nu \xi & \xi ^{2}%
\end{array}%
\right) ^{-1}. 
\]%
We note that if we multiply $\alpha ,\beta ,\gamma ,\delta $ through by a
factor $k$ (in the expression above), and multiply $\lambda ,\mu ,\nu ,\xi $
through by a factor $l$, then the image of $A$ is multiplied by a factor $%
k^{-1}l^{-1}$. So we can ignore the factor $(\alpha \delta -\beta \gamma
)^{-1}$ and still get the same orbits.

We actually have an action of GL$(2,p)\times \,$GL$(2,p)$ on the vector
space of $4\times 3$ matrices, and if we leave out the factor $(\alpha
\delta -\beta \gamma )^{-1}$ (as described above) then the kernel of the
action is the subgroup $\{(kI,kI)\,|$\thinspace $k\neq 0\}$, so (in effect)
we have a group of order $p^{2}\left( p-1\right) ^{3}\left( p+1\right) ^{2}$
acting on a space of order $p^{12}$.

If we take $\mu =0$ in the matrices above, then we obtain a subgroup $H$ of
the automorphism group of index $p+1$. There are 
\[
f(p)=p^{6}+2p^{5}+4p^{4}+8p^{3}+15p^{2}+29p+27+(2p+3)\gcd (p-1,3)
\]%
orbits of matrices under the action of $H$, and we can \textquotedblleft
write down\textquotedblright\ a set of representatives for these orbits.
However for $p=19$ this takes about 3 minutes on my 5 year old linux box,
and the representatives take up 4.5 gigabytes of space. So I save space by
not writing all the representatives down in the program to generate orbit
representatives under the action of the full group $G$.

There is a \textsc{Magma} program to compute a set of orbit representatives
under the action of the full group $G$ in notes4.1case5.m. The
representatives are stored as $4\times 3$ matrices over GF$(p)$, which takes
up less space than storing them as integer sequences. We compute a
tranversal for the subgroup $H$ in $G$, and for each of the $f(p)$ $H$-orbit
representatives $A$, we compute the images of $A$ under elements of the
transversal, and determine how the $H$-orbits fuse under the action of $G$%
.Thus we have to consider $(p+1)f(p)$ matrices $At$ where $A$ is an $H$%
-orbit representative and $t$ is an element of the transversal. For each
such matrix $At$ we compute the $H$-orbit representative of $At$. (This
takes a bounded amount of work involving arithmetic over GF$(p)$.) We index
the $H$-orbits, and we add an $H$-orbit representative $A$ to the list of
the $G$-orbit representatives if the index of the $H$-orbit containing $A$
is greater than or equal to the indexes of the $H$-orbits containing the
matrices $At$ for $t$ in the transversal. So, if the index of the $H$-orbit
containing $At$ is less than the index of the $H$-orbit containing $A$, then
we discard $A$ and there is no need to consider the elements $Au$ for $u$ in
the remainder of the transversal. This means that we don't actually have to
consider all the elements $At$. For $p=3$ we only need to consider less than
two thirds of the elements $At$, for $p=5$ less than a half, for $p=7$ a
little over a third, and so on. Experimentally, it seems that the proportion
drops as the prime increases. So the total amount of work needed to compute
a set of representatives for the $G$-orbits is of order somewhere between $%
p^{6}$ and $p^{7}$. For $p\leq 23$ the time taken for the program to run is
roughly proportional to $p^{6.2}$. However this is a serious bottleneck, and
it takes about two hours to generate the list for $p=19$ on my five year old
linux box. Note however that $19^{5}=\allowbreak 2476\,099$, and there is
probably only a limited amount of interesting work you can do with two and
half million groups of order $19^{7}$.

\subsection{Case 6}

Let $L$ satisfy $da=0,\,db=\omega ca,\,dc=ba$. Then $L^{2}$ is generated by $%
ba$, $ca$, $cb$ and $pL\leq L^{2}$. It is straightforward to show that all
elements in the linear span of $a,b,c,d$ have breadth 3, except for those of
the form $\alpha a+\delta d$. Using this we can show that if $a^{\prime
},b^{\prime },c^{\prime },d^{\prime }$ generate $L$ and satisfy the same
commutator relations as $a,b,c,d$ then (modulo $L^{2}$) 
\begin{eqnarray*}
a^{\prime } &=&\alpha a+\delta d, \\
b^{\prime } &=&\pm (\lambda a+\gamma b+\omega \beta c+\mu d), \\
c^{\prime } &=&\nu a+\beta b+\gamma c+\xi d, \\
d^{\prime } &=&\pm (\omega \delta a+\alpha d)
\end{eqnarray*}%
and 
\[
\left( 
\begin{array}{c}
b^{\prime }a^{\prime } \\ 
c^{\prime }a^{\prime } \\ 
c^{\prime }b^{\prime }%
\end{array}%
\right) =\left( 
\begin{array}{ccc}
\pm (\alpha \gamma -\omega \beta \delta ) & \pm (\omega \alpha \beta -\omega
\gamma \delta ) & 0 \\ 
\alpha \beta -\gamma \delta & \alpha \gamma -\omega \beta \delta & 0 \\ 
\pm (\beta \lambda -\gamma \nu +\omega \beta \xi -\gamma \mu ) & \pm (\gamma
\lambda -\omega \beta \mu +\omega \gamma \xi -\omega \beta \nu ) & \pm
(\gamma ^{2}-\omega \beta ^{2})%
\end{array}%
\right) \left( 
\begin{array}{c}
ba \\ 
ca \\ 
cb%
\end{array}%
\right) . 
\]%
We let 
\[
\left( 
\begin{array}{c}
pa \\ 
pb \\ 
pc \\ 
pd%
\end{array}%
\right) =A\left( 
\begin{array}{c}
ba \\ 
ca \\ 
cb%
\end{array}%
\right) 
\]%
where $A$ is a $4\times 3$ matrix over $\mathbb{Z}_{p}$. Then under a change
of generating set of the form described above we see that 
\[
A\mapsto \left( 
\begin{array}{cccc}
\alpha & 0 & 0 & \delta \\ 
\pm \lambda & \pm \gamma & \pm \omega \beta & \pm \mu \\ 
\nu & \beta & \gamma & \xi \\ 
\pm \omega \delta & 0 & 0 & \pm \alpha%
\end{array}%
\right) AB^{-1}, 
\]%
where%
\[
B=\left( 
\begin{array}{ccc}
\pm (\alpha \gamma -\omega \beta \delta ) & \pm (\omega \alpha \beta -\omega
\gamma \delta ) & 0 \\ 
\alpha \beta -\gamma \delta & \alpha \gamma -\omega \beta \delta & 0 \\ 
\pm (\beta \lambda -\gamma \nu +\omega \beta \xi -\gamma \mu ) & \pm (\gamma
\lambda -\omega \beta \mu +\omega \gamma \xi -\omega \beta \nu ) & \pm
(\gamma ^{2}-\omega \beta ^{2})%
\end{array}%
\right) . 
\]

We note that $\langle a,d\rangle +L^{2}$ is a characteristic subalgebra, and
first investigate the orbits of $pa,pd$. We consider three separate cases: $%
pa=pd=0$, $pa$ and $pd$ span a one dimensional subspace, and $pa,pd$ are
linearly independent. It turns out that there are $p+4$ orbits of $pa,pd$.
It is quite easy to see that if $pa,pd$ do \emph{not} span $\langle
ba,ca\rangle $ then we can assume that $pa=pd=0$, or $pa=0$, $pd=ca$, or $%
pa=0$, $pd=cb$, or $pa=ca$, $pd=cb$. There are $p$ orbits where $pa,pd$ span 
$\langle ba,ca\rangle $, and we have a \textsc{Magma} program to find them.

\subsubsection{$pa=pd=0$}

If $pb,pc$ don't both lie in $\langle ba,ca\rangle $ then we can take $pb\in
\langle ba,ca\rangle $ and $pc\notin \langle ba,ca\rangle $, which mean we
need to take $\beta =0$. We can then take $pc=cb$, which means we need to
take $\gamma =1$ in the $+$ matrices and $\gamma =-1$ in the $-$ matrices.
We can then take $pc=0$ or $ca$. There are $p$ orbits when $pb,pc\in \langle
ba,ca\rangle $, and there is a \textsc{Magma} program to find them.

\subsubsection{$pa=0$, $pd=ca$}

We need $\delta =0$, $\beta =0$ in both the plus and minus matrices, and $%
\gamma =1$ in the plus matrices and $\gamma =-1$ in the minus matrices. We
then have:

\[
\left( 
\begin{array}{cccc}
\alpha & 0 & 0 & \delta \\ 
\lambda & \gamma & \omega \beta & \mu \\ 
\nu & \beta & \gamma & \xi \\ 
\omega \delta & 0 & 0 & \alpha%
\end{array}%
\right) \left( 
\begin{array}{ccc}
0 & 0 & 0 \\ 
u & v & w \\ 
x & y & z \\ 
0 & 1 & 0%
\end{array}%
\right) \left( 
\begin{array}{ccc}
(\alpha \gamma -\omega \beta \delta ) & (\omega \alpha \beta -\omega \gamma
\delta ) & 0 \\ 
\alpha \beta -\gamma \delta & \alpha \gamma -\omega \beta \delta & 0 \\ 
(\beta \lambda -\gamma \nu +\omega \beta \xi -\gamma \mu ) & (\gamma \lambda
-\omega \beta \mu +\omega \gamma \xi -\omega \beta \nu ) & (\gamma
^{2}-\omega \beta ^{2})%
\end{array}%
\right) ^{-1} 
\]
\[
=\left( 
\begin{array}{ccc}
0 & 0 & 0 \\ 
\frac{1}{\alpha }\left( u+w\mu +w\nu \right) & \frac{1}{\alpha }\left( v+\mu
-w\lambda -w\xi \omega \right) & w \\ 
\frac{1}{\alpha }\left( x+z\mu +z\nu \right) & \frac{1}{\alpha }\left( y+\xi
-z\lambda -z\xi \omega \right) & z \\ 
0 & 1 & 0%
\end{array}%
\right) \allowbreak 
\]

\[
\left( 
\begin{array}{cccc}
\alpha & 0 & 0 & \delta \\ 
-\lambda & -\gamma & -\omega \beta & -\mu \\ 
\nu & \beta & \gamma & \xi \\ 
-\omega \delta & 0 & 0 & -\alpha%
\end{array}%
\right) \left( 
\begin{array}{ccc}
0 & 0 & 0 \\ 
u & v & w \\ 
x & y & z \\ 
0 & 1 & 0%
\end{array}%
\right) \left( 
\begin{array}{ccc}
-(\alpha \gamma -\omega \beta \delta ) & -(\omega \alpha \beta -\omega
\gamma \delta ) & 0 \\ 
\alpha \beta -\gamma \delta & \alpha \gamma -\omega \beta \delta & 0 \\ 
-(\beta \lambda -\gamma \nu +\omega \beta \xi -\gamma \mu ) & -(\gamma
\lambda -\omega \beta \mu +\omega \gamma \xi -\omega \beta \nu ) & -(\gamma
^{2}-\omega \beta ^{2})%
\end{array}%
\right) ^{-1} 
\]
\[
=\left( 
\begin{array}{ccc}
0 & 0 & 0 \\ 
-\frac{1}{\alpha }\left( w\mu -u+w\nu \right) & -\frac{1}{\alpha }\left(
v-\mu +w\lambda +w\xi \omega \right) & -w \\ 
\frac{1}{\alpha }\left( z\mu -x+z\nu \right) & \frac{1}{\alpha }\left( y-\xi
+z\lambda +z\xi \omega \right) & z \\ 
0 & 1 & 0%
\end{array}%
\right) \allowbreak 
\]

We can assume that $0\leq w\leq (p-1)/2$. If $w\neq 0$ we can assume that $%
u=v=y=0$, that $x=0$ or $1$, with no restriction on $z$.

If $w=0$ and $z\neq 0$, we can assume that $u=0$ or $1$, and that $v=x=y=0$.

If $w=z=0$, we can assume that $v=y=0$, and that $u=0$ and $x=0$ or $1$, or
that $u=1$ and $0\leq x\leq (p-1)/2$.

\subsubsection{ $pa=0$, $pd=cb$}

We need $\delta =0$, $\lambda =-\xi \omega $, $\mu =-\nu $, $\alpha =\gamma
^{2}-\beta ^{2}\omega $ in both plus and minus matrices, giving: 
\[
\left( 
\begin{array}{cccc}
\alpha & 0 & 0 & \delta \\ 
\lambda & \gamma & \omega \beta & \mu \\ 
\nu & \beta & \gamma & \xi \\ 
\omega \delta & 0 & 0 & \alpha%
\end{array}%
\right) \left( 
\begin{array}{ccc}
0 & 0 & 0 \\ 
u & v & w \\ 
x & y & z \\ 
0 & 0 & 1%
\end{array}%
\right) \left( 
\begin{array}{ccc}
(\alpha \gamma -\omega \beta \delta ) & (\omega \alpha \beta -\omega \gamma
\delta ) & 0 \\ 
\alpha \beta -\gamma \delta & \alpha \gamma -\omega \beta \delta & 0 \\ 
(\beta \lambda -\gamma \nu +\omega \beta \xi -\gamma \mu ) & (\gamma \lambda
-\omega \beta \mu +\omega \gamma \xi -\omega \beta \nu ) & (\gamma
^{2}-\omega \beta ^{2})%
\end{array}%
\right) ^{-1} 
\]
\[
=\left( 
\begin{array}{ccc}
0 & 0 & 0 \\ 
\frac{1}{\left( \gamma ^{2}-\beta ^{2}\omega \right) ^{2}}\left( u\gamma
^{2}-v\beta \gamma -y\beta ^{2}\omega +x\beta \gamma \omega \right) & \frac{1%
}{\left( \gamma ^{2}-\beta ^{2}\omega \right) ^{2}}\left( v\gamma
^{2}-x\beta ^{2}\omega ^{2}-u\beta \gamma \omega +y\beta \gamma \omega
\right) & \frac{1}{\gamma ^{2}-\beta ^{2}\omega }\left( w\gamma -\nu +z\beta
\omega \right) \\ 
-\frac{1}{\left( \gamma ^{2}-\beta ^{2}\omega \right) ^{2}}\left( v\beta
^{2}-x\gamma ^{2}-u\beta \gamma +y\beta \gamma \right) & \frac{1}{\left(
\gamma ^{2}-\beta ^{2}\omega \right) ^{2}}\left( y\gamma ^{2}+v\beta \gamma
-u\beta ^{2}\omega -x\beta \gamma \omega \right) & \frac{1}{\gamma
^{2}-\beta ^{2}\omega }\left( \xi +w\beta +z\gamma \right) \\ 
0 & 0 & 1%
\end{array}%
\right) \allowbreak 
\]

\[
\left( 
\begin{array}{cccc}
\alpha & 0 & 0 & \delta \\ 
-\lambda & -\gamma & -\omega \beta & -\mu \\ 
\nu & \beta & \gamma & \xi \\ 
-\omega \delta & 0 & 0 & -\alpha%
\end{array}%
\right) \left( 
\begin{array}{ccc}
0 & 0 & 0 \\ 
u & v & w \\ 
x & y & z \\ 
0 & 0 & 1%
\end{array}%
\right) \left( 
\begin{array}{ccc}
-(\alpha \gamma -\omega \beta \delta ) & -(\omega \alpha \beta -\omega
\gamma \delta ) & 0 \\ 
\alpha \beta -\gamma \delta & \alpha \gamma -\omega \beta \delta & 0 \\ 
-(\beta \lambda -\gamma \nu +\omega \beta \xi -\gamma \mu ) & -(\gamma
\lambda -\omega \beta \mu +\omega \gamma \xi -\omega \beta \nu ) & -(\gamma
^{2}-\omega \beta ^{2})%
\end{array}%
\right) ^{-1} 
\]
\[
=\left( 
\begin{array}{ccc}
0 & 0 & 0 \\ 
\frac{1}{\left( \gamma ^{2}-\beta ^{2}\omega \right) ^{2}}\left( u\gamma
^{2}-v\beta \gamma -y\beta ^{2}\omega +x\beta \gamma \omega \right) & -\frac{%
1}{\left( \gamma ^{2}-\beta ^{2}\omega \right) ^{2}}\left( v\gamma
^{2}-x\beta ^{2}\omega ^{2}-u\beta \gamma \omega +y\beta \gamma \omega
\right) & \frac{1}{\gamma ^{2}-\beta ^{2}\omega }\left( w\gamma -\nu +z\beta
\omega \right) \\ 
\frac{1}{\left( \gamma ^{2}-\beta ^{2}\omega \right) ^{2}}\left( v\beta
^{2}-x\gamma ^{2}-u\beta \gamma +y\beta \gamma \right) & \frac{1}{\left(
\gamma ^{2}-\beta ^{2}\omega \right) ^{2}}\left( y\gamma ^{2}+v\beta \gamma
-u\beta ^{2}\omega -x\beta \gamma \omega \right) & -\frac{1}{\gamma
^{2}-\beta ^{2}\omega }\left( \xi +w\beta +z\gamma \right) \\ 
0 & 0 & 1%
\end{array}%
\right) \allowbreak 
\]

So we can take $w=z=0$, and we can assume that $u=0,1$, or the least
non-square. (Experimentally only 0 and 1 arise, but I don't have a proof of
this.) There is a \textsc{Magma} program to find the orbits of $u,v,x,y$.

\subsubsection{$pa=ca$, $pd=cb$}

We need $\delta =0$, $\beta =0$ and $\gamma =1$ in both the plus and minus
matrices. You also need $\lambda =-\xi \omega $, $\mu =-\nu $, and $\alpha
=1 $. We then have:

\[
\left( 
\begin{array}{cccc}
\alpha & 0 & 0 & \delta \\ 
\lambda & \gamma & \omega \beta & \mu \\ 
\nu & \beta & \gamma & \xi \\ 
\omega \delta & 0 & 0 & \alpha%
\end{array}%
\right) \left( 
\begin{array}{ccc}
0 & 1 & 0 \\ 
u & v & w \\ 
x & y & z \\ 
0 & 0 & 1%
\end{array}%
\right) \left( 
\begin{array}{ccc}
(\alpha \gamma -\omega \beta \delta ) & (\omega \alpha \beta -\omega \gamma
\delta ) & 0 \\ 
\alpha \beta -\gamma \delta & \alpha \gamma -\omega \beta \delta & 0 \\ 
(\beta \lambda -\gamma \nu +\omega \beta \xi -\gamma \mu ) & (\gamma \lambda
-\omega \beta \mu +\omega \gamma \xi -\omega \beta \nu ) & (\gamma
^{2}-\omega \beta ^{2})%
\end{array}%
\right) ^{-1} 
\]
\[
=\left( 
\begin{array}{ccc}
0 & 1 & 0 \\ 
u & v-\xi \omega & w-\nu \\ 
x & y+\nu & z+\xi \\ 
0 & 0 & 1%
\end{array}%
\right) 
\]

\[
\left( 
\begin{array}{cccc}
\alpha & 0 & 0 & \delta \\ 
-\lambda & -\gamma & -\omega \beta & -\mu \\ 
\nu & \beta & \gamma & \xi \\ 
-\omega \delta & 0 & 0 & -\alpha%
\end{array}%
\right) \left( 
\begin{array}{ccc}
0 & 1 & 0 \\ 
u & v & w \\ 
x & y & z \\ 
0 & 0 & 1%
\end{array}%
\right) \left( 
\begin{array}{ccc}
-(\alpha \gamma -\omega \beta \delta ) & -(\omega \alpha \beta -\omega
\gamma \delta ) & 0 \\ 
\alpha \beta -\gamma \delta & \alpha \gamma -\omega \beta \delta & 0 \\ 
-(\beta \lambda -\gamma \nu +\omega \beta \xi -\gamma \mu ) & -(\gamma
\lambda -\omega \beta \mu +\omega \gamma \xi -\omega \beta \nu ) & -(\gamma
^{2}-\omega \beta ^{2})%
\end{array}%
\right) ^{-1} 
\]
\[
=\left( 
\begin{array}{ccc}
0 & 1 & 0 \\ 
u & \xi \omega -v & w-\nu \\ 
-x & y+\nu & -z-\xi \\ 
0 & 0 & 1%
\end{array}%
\right) 
\]

So you can take $v=w=0$ and $0\leq x\leq (p-1)/2$. If $x=0$ you can take $%
0\leq z\leq (p-1)/2$.

\subsubsection{$pa,pd$ span $\langle ba,ca\rangle $}

If $pb,pc$ both lie in $\langle ba,ca\rangle $, then we can assume that $%
pb=pc=0$, and that $pa=ca$. There is a \textsc{Magma} program to find the $p$
orbits of $pd$.

If $pb,pc$ don't both lie in $\langle ba,ca\rangle $, then we can assume
that $pb=0$, and that $pc\in \langle ba,ca\rangle +cb$ though we then need $%
\beta =0$, and $\gamma =1$ in the plus matrices and $\gamma =-1$ in the
minus matrices. This gives:

\[
\left( 
\begin{array}{cccc}
\alpha & 0 & 0 & \delta \\ 
\lambda & \gamma & \omega \beta & \mu \\ 
\nu & \beta & \gamma & \xi \\ 
\omega \delta & 0 & 0 & \alpha%
\end{array}%
\right) \left( 
\begin{array}{ccc}
u & v & 0 \\ 
0 & 0 & 0 \\ 
x & y & 1 \\ 
p & q & 0%
\end{array}%
\right) \left( 
\begin{array}{ccc}
(\alpha \gamma -\omega \beta \delta ) & (\omega \alpha \beta -\omega \gamma
\delta ) & 0 \\ 
\alpha \beta -\gamma \delta & \alpha \gamma -\omega \beta \delta & 0 \\ 
(\beta \lambda -\gamma \nu +\omega \beta \xi -\gamma \mu ) & (\gamma \lambda
-\omega \beta \mu +\omega \gamma \xi -\omega \beta \nu ) & (\gamma
^{2}-\omega \beta ^{2})%
\end{array}%
\right) ^{-1} 
\]
\[
=\left( 
\begin{array}{ccc}
\frac{1}{\alpha ^{2}-\delta ^{2}\omega }\left( q\delta ^{2}+u\alpha
^{2}+p\alpha \delta +v\alpha \delta \right) & \frac{1}{\alpha ^{2}-\delta
^{2}\omega }\left( v\alpha ^{2}+q\alpha \delta +p\delta ^{2}\omega +u\alpha
\delta \omega \right) & 0 \\ 
\frac{1}{\alpha ^{2}-\delta ^{2}\omega }\left( p\alpha \mu +q\mu \delta
+u\alpha \lambda +v\lambda \delta \right) & \frac{1}{\alpha ^{2}-\delta
^{2}\omega }\left( q\alpha \mu +v\alpha \lambda +p\mu \delta \omega
+u\lambda \delta \omega \right) & 0 \\ 
\frac{1}{\alpha ^{2}-\delta ^{2}\omega }\left( x\alpha +y\delta +\alpha \mu
+\alpha \nu -\lambda \delta +p\alpha \xi +q\delta \xi +u\alpha \nu +v\delta
\nu -\delta \xi \omega \right) & \frac{1}{\alpha ^{2}-\delta ^{2}\omega }%
\left( y\alpha -\alpha \lambda +q\alpha \xi +v\alpha \nu +x\delta \omega
-\alpha \xi \omega +\mu \delta \omega +\delta \nu \omega +p\delta \xi \omega
+u\delta \nu \omega \right) & 1 \\ 
\frac{1}{\alpha ^{2}-\delta ^{2}\omega }\left( p\alpha ^{2}+q\alpha \delta
+v\delta ^{2}\omega +u\alpha \delta \omega \right) & \frac{1}{\alpha
^{2}-\delta ^{2}\omega }\left( q\alpha ^{2}+u\delta ^{2}\omega ^{2}+p\alpha
\delta \omega +v\alpha \delta \omega \right) & 0%
\end{array}%
\right) \allowbreak 
\]

\[
\left( 
\begin{array}{cccc}
\alpha & 0 & 0 & \delta \\ 
-\lambda & -\gamma & -\omega \beta & -\mu \\ 
\nu & \beta & \gamma & \xi \\ 
-\omega \delta & 0 & 0 & -\alpha%
\end{array}%
\right) \left( 
\begin{array}{ccc}
u & v & 0 \\ 
0 & 0 & 0 \\ 
x & y & 1 \\ 
p & q & 0%
\end{array}%
\right) \left( 
\begin{array}{ccc}
-(\alpha \gamma -\omega \beta \delta ) & -(\omega \alpha \beta -\omega
\gamma \delta ) & 0 \\ 
\alpha \beta -\gamma \delta & \alpha \gamma -\omega \beta \delta & 0 \\ 
-(\beta \lambda -\gamma \nu +\omega \beta \xi -\gamma \mu ) & -(\gamma
\lambda -\omega \beta \mu +\omega \gamma \xi -\omega \beta \nu ) & -(\gamma
^{2}-\omega \beta ^{2})%
\end{array}%
\right) ^{-1} 
\]
\[
\left( 
\begin{array}{ccc}
\frac{1}{\alpha ^{2}-\delta ^{2}\omega }\left( q\delta ^{2}+u\alpha
^{2}+p\alpha \delta +v\alpha \delta \right) & -\frac{1}{\alpha ^{2}-\delta
^{2}\omega }\left( v\alpha ^{2}+q\alpha \delta +p\delta ^{2}\omega +u\alpha
\delta \omega \right) & 0 \\ 
-\frac{1}{\alpha ^{2}-\delta ^{2}\omega }\left( p\alpha \mu +q\mu \delta
+u\alpha \lambda +v\lambda \delta \right) & \frac{1}{\alpha ^{2}-\delta
^{2}\omega }\left( q\alpha \mu +v\alpha \lambda +p\mu \delta \omega
+u\lambda \delta \omega \right) & 0 \\ 
\frac{1}{\alpha ^{2}-\delta ^{2}\omega }\left( \alpha \mu -y\delta -x\alpha
+\alpha \nu -\lambda \delta +p\alpha \xi +q\delta \xi +u\alpha \nu +v\delta
\nu -\delta \xi \omega \right) & -\frac{1}{\alpha ^{2}-\delta ^{2}\omega }%
\left( q\alpha \xi -\alpha \lambda -y\alpha +v\alpha \nu -x\delta \omega
-\alpha \xi \omega +\mu \delta \omega +\delta \nu \omega +p\delta \xi \omega
+u\delta \nu \omega \right) & 1 \\ 
-\frac{1}{\alpha ^{2}-\delta ^{2}\omega }\left( p\alpha ^{2}+q\alpha \delta
+v\delta ^{2}\omega +u\alpha \delta \omega \right) & \frac{1}{\alpha
^{2}-\delta ^{2}\omega }\left( q\alpha ^{2}+u\delta ^{2}\omega ^{2}+p\alpha
\delta \omega +v\alpha \delta \omega \right) & 0%
\end{array}%
\right) \allowbreak 
\]

So we need $\lambda =0$, $\mu =0$ giving

\[
\left( 
\begin{array}{cccc}
\alpha & 0 & 0 & \delta \\ 
\lambda & \gamma & \omega \beta & \mu \\ 
\nu & \beta & \gamma & \xi \\ 
\omega \delta & 0 & 0 & \alpha%
\end{array}%
\right) \left( 
\begin{array}{ccc}
u & v & 0 \\ 
0 & 0 & 0 \\ 
x & y & 1 \\ 
p & q & 0%
\end{array}%
\right) \left( 
\begin{array}{ccc}
(\alpha \gamma -\omega \beta \delta ) & (\omega \alpha \beta -\omega \gamma
\delta ) & 0 \\ 
\alpha \beta -\gamma \delta & \alpha \gamma -\omega \beta \delta & 0 \\ 
(\beta \lambda -\gamma \nu +\omega \beta \xi -\gamma \mu ) & (\gamma \lambda
-\omega \beta \mu +\omega \gamma \xi -\omega \beta \nu ) & (\gamma
^{2}-\omega \beta ^{2})%
\end{array}%
\right) ^{-1} 
\]
\[
=\left( 
\begin{array}{ccc}
\frac{1}{\alpha ^{2}-\delta ^{2}\omega }\left( q\delta ^{2}+u\alpha
^{2}+p\alpha \delta +v\alpha \delta \right) & \frac{1}{\alpha ^{2}-\delta
^{2}\omega }\left( v\alpha ^{2}+q\alpha \delta +p\delta ^{2}\omega +u\alpha
\delta \omega \right) & 0 \\ 
0 & 0 & 0 \\ 
\frac{1}{\alpha ^{2}-\delta ^{2}\omega }\left( x\alpha +y\delta +\alpha \nu
+p\alpha \xi +q\delta \xi +u\alpha \nu +v\delta \nu -\delta \xi \omega
\right) & \frac{1}{\alpha ^{2}-\delta ^{2}\omega }\left( y\alpha +q\alpha
\xi +v\alpha \nu +x\delta \omega -\alpha \xi \omega +\delta \nu \omega
+p\delta \xi \omega +u\delta \nu \omega \right) & 1 \\ 
\frac{1}{\alpha ^{2}-\delta ^{2}\omega }\left( p\alpha ^{2}+q\alpha \delta
+v\delta ^{2}\omega +u\alpha \delta \omega \right) & \frac{1}{\alpha
^{2}-\delta ^{2}\omega }\left( q\alpha ^{2}+u\delta ^{2}\omega ^{2}+p\alpha
\delta \omega +v\alpha \delta \omega \right) & 0%
\end{array}%
\right) \allowbreak 
\]

\[
\left( 
\begin{array}{cccc}
\alpha & 0 & 0 & \delta \\ 
-\lambda & -\gamma & -\omega \beta & -\mu \\ 
\nu & \beta & \gamma & \xi \\ 
-\omega \delta & 0 & 0 & -\alpha%
\end{array}%
\right) \left( 
\begin{array}{ccc}
u & v & 0 \\ 
0 & 0 & 0 \\ 
x & y & 1 \\ 
p & q & 0%
\end{array}%
\right) \left( 
\begin{array}{ccc}
-(\alpha \gamma -\omega \beta \delta ) & -(\omega \alpha \beta -\omega
\gamma \delta ) & 0 \\ 
\alpha \beta -\gamma \delta & \alpha \gamma -\omega \beta \delta & 0 \\ 
-(\beta \lambda -\gamma \nu +\omega \beta \xi -\gamma \mu ) & -(\gamma
\lambda -\omega \beta \mu +\omega \gamma \xi -\omega \beta \nu ) & -(\gamma
^{2}-\omega \beta ^{2})%
\end{array}%
\right) ^{-1} 
\]
\[
=\left( 
\begin{array}{ccc}
\frac{1}{\alpha ^{2}-\delta ^{2}\omega }\left( q\delta ^{2}+u\alpha
^{2}+p\alpha \delta +v\alpha \delta \right) & -\frac{1}{\alpha ^{2}-\delta
^{2}\omega }\left( v\alpha ^{2}+q\alpha \delta +p\delta ^{2}\omega +u\alpha
\delta \omega \right) & 0 \\ 
0 & 0 & 0 \\ 
\frac{1}{\alpha ^{2}-\delta ^{2}\omega }\left( \alpha \nu -y\delta -x\alpha
+p\alpha \xi +q\delta \xi +u\alpha \nu +v\delta \nu -\delta \xi \omega
\right) & -\frac{1}{\alpha ^{2}-\delta ^{2}\omega }\left( q\alpha \xi
-y\alpha +v\alpha \nu -x\delta \omega -\alpha \xi \omega +\delta \nu \omega
+p\delta \xi \omega +u\delta \nu \omega \right) & 1 \\ 
-\frac{1}{\alpha ^{2}-\delta ^{2}\omega }\left( p\alpha ^{2}+q\alpha \delta
+v\delta ^{2}\omega +u\alpha \delta \omega \right) & \frac{1}{\alpha
^{2}-\delta ^{2}\omega }\left( q\alpha ^{2}+u\delta ^{2}\omega ^{2}+p\alpha
\delta \omega +v\alpha \delta \omega \right) & 0%
\end{array}%
\right) \allowbreak 
\]

Note that the values of $pa$ and $pd$ depend only on $\alpha ,\delta $
(together with their original values), and that replacing $\alpha ,\delta $
by $\alpha k,\delta k$ makes no difference. There is a \textsc{Magma}
program to compute the orbits of $pa,pd$ under this action. It isn't
particularly easy to see, but for any fixed values of $pa,pd$, we can always
take $x=0$, and $y=0$ or 1. Just to make things tricky, for some fixed $%
pa,pd $, $x=y=0$ is in the same orbit as $x=0$, $y=1$, and sometimes it
isn't. There is a \textsc{Magma} program, notes4.1case6.m, to sort this out.

\end{document}
